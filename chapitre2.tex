\chapter{Formulation relativiste}


{\small \it Note : la première partie se trouve sur une feuille distribuée par le professeur.}
\section{Rappels d'analyse tensorielle}
\section{Formulation relativiste de l'électrodynamique}
\subsection{Postulats fondamentaux}

\begin{postulat}
	Les lois de la physique, lorsqu'elles sont formulées de manière adéquate, gardent la m\^eme forme dans un référentiel $\mathcal{R}$ et dans un référentiel $\mathcal{R}'$ en translation rectiligne uniforme par rapport à $\mathcal{R}$. Il y a invariance de la forme des équations.\\
	Contribution d'Einstein : la vitesse de propagation d'un signal électromagnétique dans le vide est universelle et indépendante du référentiel.
\end{postulat}

\begin{cons}
	Ce sont les équations de Maxwell qui sont invariantes, et cela a donc amené à abandonner la mécanique classique (Newtonienne) pour la reformuler afin qu'elle soit aussi invariante sous une transformation de Lorentz.
\end{cons}

\begin{postulat}
	Invariance de la charge : en accord avec les expériences, on postule que la charge q d'une particule ne dépend pas du référentiel galiléen considéré.
\end{postulat}

\begin{cons}
	L'invariance formelle des lois de la physique est automatiquement satisfaite si on les écrit sous forme temporelle (condition suffisante), sachant que les transformations à considérer sont les transformations du groupe de Lorentz, qui recouvre les translations spatiales et temporelles, les rotations, les renversements d'axe, et les boosts de Lorentz.
$$
	x'^{\mu}=\mathcal{L}^{\mu}_{\nu}x^{\mu} \text{ avec }\mathcal{L}^{\mu}_{\nu}=\begin{mat}
	\gamma & -\beta\gamma & \hspace*{0.2cm}0\hspace*{0.2cm} \\
	-\beta\gamma & \gamma & \hspace*{0.2cm}0\hspace*{0.2cm} \\
	0 & 0 & \hspace*{0.2cm}1\hspace*{0.2cm}
	\end{mat}
$$
\end{cons}


\subsection{Quadrivecteur courant}
{\txt
On cherche à construire un quadrivecteur à partir de $\rho$ et $\vec{j}$. Ainsi, on considère une charge $q$ dans un référentiel $\mathcal{R}$, que l'on modélise comme une distribution de charge $\rho$ dans un volume $dV=\frac{q}{\rho}$. Dans $\mathcal{R}'$ en translation rectiligne uniforme par rapport à $\mathcal{R}$, cette charge $q$ occupe $dV'$ et correspond à $\rho'$. D'après le postulat d'invariance de la charge, on a : $\rho dV=q=\rho' dV'$. Dans $\mathcal{R}$, la charge est en mouvement et est donc associée à un vecteur densité de courant :}
$$
	\vec{j}=\begin{cases}
		\rho\vec{v}&\text{si }\vec{r}\in dV\\
		0&\text{sinon}
	\end{cases}
$$

{\txt
Pendant un intervalle de temps $dt$ dans $\mathcal{R}$, la charge se déplace de $dx^\mu=(dt,d\vec{r})$. $dx^\mu$ étant un quadrivecteur, il en est de m\^eme pour $\rho dVdx^\mu$. On considère alors $\rho\frac{cdtdV}{c}\frac{dx^\mu}{dt}$}.

\begin{postulat}
	$cdtdV$ est un invariant.
\end{postulat}

\begin{proof}
	$cdtdV$ étant l'élément d'intégration dans l'espace de Minkovski, on considère la matrice jacobienne.
\begin{rappel}
	Soit $\varphi:(x,y,z)\longrightarrow \vec{\varphi}(x,y,z)=\begin{mat}
		u(x,y,z)\\
		v\\
		w
	\end{mat}$
	
	$$
		\text{Jac}_\varphi=\frac{D(u,v,w)}{D(x,y,z)}=\begin{mat}
			\partial_xu & \partial_yu & \partial_zu\\
			\partial_xv & \partial_yv & \partial_zv\\
			\partial_xw & \partial_yw & \partial_zw
		\end{mat}\\
	$$
	$$
		J_\varphi=\det(\text{Jac}_\varphi)
	$$
	$$
		\text{alors }\bigintssss_{\varphi(x)}f(u_1,...u_n)\,du_1...du_n=\bigintssss_Xf\circ\varphi(x_1,...x_n)|J_\varphi|\,dx_1...dx_n
	$$
\end{rappel}
Pour un boost de Lorentz, on a :
$$
	\begin{mat}
		ct' \\ x' \\ t' \\ z'
	\end{mat}
	=
	\begin{mat}
		\gamma & -\beta\gamma & 0 & 0\\
		-\beta\gamma & \gamma & 0 & 0\\
		0 & 0 & 1 & 0\\
		0 & 0 & 0 & 1
	\end{mat}
	\begin{mat}
		ct \\ x \\ y \\ z
	\end{mat}
	\text{ et par conséquent }J_{boost}=1
$$
$$
	\begin{array}{r@{\;}l}
		\text{d'où\hspace*{0.5cm}} cdtdxdydz&=cdt'dx'dy'dz'\\
		cdtdV&=cdt'dV'
	\end{array}
$$
\end{proof}

\begin{remark}
{\txt Attention aux démonstrations avec $dt'=\gamma dt$ et $dx'=\frac{1}{\gamma}dx$}
\end{remark}

\begin{corol}
	{\txt $cdtdxdydz$ est invariant et donc $\rho\frac{dx^\mu}{dt}$ est un quadrivecteur.}\\
	On pose $J^\mu=\rho\frac{dx^\mu}{dt}$ le quadrivecteur courant (composantes contravariantes).\\
	$J^\mu=(\rho c,\vec{j})=(\rho c,\rho \vec{v})=\rho_0(\gamma c,\gamma\vec{v})$ où $\rho_0$ est la densité de charges dans le référentiel propre de la particule.
\end{corol}

\subsection{Formulation covariante de l'électromagnétisme}
\subsubsection{Retour sur un opérateur}

On a vu que $\nabla$ est un quadrivecteur dont les composantes covariantes sont $\partial_a=\frac{\partial}{\partial{x^a}}$ soit :
$$ 
	\nabla=\left(\frac{1}{c}\frac{\partial}{\partial t},\frac{\partial}{\partial x},\frac{\partial}{\partial y},\frac{\partial}{\partial z}\right)
$$
et ses composantes contravariantes s'écrivent : $\left(\frac{1}{c}\frac{\partial}{\partial t},-\frac{\partial}{\partial x},-\frac{\partial}{\partial y},-\frac{\partial}{\partial z}\right)$\\
$\partial^a\partial_a$ est un tenseur d'ordre 0 obtenu par contraction de deux tenseurs de rang 1 et est donc invariant.
$$
	\begin{array}{r@{\;}l}
			\partial^a\partial_a&=\partial^0\partial_0+\partial^1\partial_1+\partial^2\partial_2+\partial^3\partial_3\\
		&=\frac{1}{c^2}\frac{\partial^2}{\partial t^2}-\frac{\partial^2}{\partial x^2}-\frac{\partial^2}{\partial y^2}-\frac{\partial^2}{\partial z^2} = -\square
	\end{array}	
$$

\subsubsection{Quadrivecteur potentiel}
En jauge de Lorenz ($\nab\cdot\vec{A}+\frac{1}{c^2}\partial_t\phi=0$), les équations satisfaites par les potentiels sont :
$$
	\left\{ \begin{array}{r@{\;}l}
		\nab^2\phi - \frac{1}{c^2}\partial_t^2\phi&=-\frac{\rho}{\epsilon_0}\\
		\nab^2\vec{A}-\frac{1}{c^2}\partial_t^2\vec{A}&=-\mu_0\vec{j}\\
		\square\,\frac{\phi}{c}&=-\frac{\rho}{c\epsilon_0}=-\mu_0\rho c\\
		\square\,\vec{A}&=-\mu_0\vec{j}
	\end{array} \right.
$$
{\txt On est amené à considérer $(\frac{\phi}{c},\vec{A})$ comme les composantes contravariantes d'un quadrivecteur, le quadripotentiel.}
$$
	\square\,\phi^\mu=-\mu_0J^\mu
	\text{ avec } \phi^\mu=(\frac{\phi}{c},\vec{A})
$$
En ce qui concerne la jauge de Lorenz,
$$
	\begin{array}{r@{\;}l}
		0&=\nab\cdot\vec{A}+\frac{1}{c^2}\partial_t\phi=\partial_\mu\phi^\mu\\
		&=\partial_0\phi^0+\partial_1\phi^1+\partial_2\phi^2+\partial_3\phi^3\\
		&=\frac{1}{c}\partial_t\left(\frac{\phi}{c}\right)+\underbrace{\partial_xA_x+\partial_yA_y+\partial_zA_z}_{\nab\cdot\vec{A}}
	\end{array}
$$
La jauge est covariante.

\begin{conc}
	On a été capable de reformuler l'électromagnétique sous forme covariante :
	$$
		\left\{ \begin{array}{r@{\;}l}
			\partial_\mu\phi^\mu&=0\text{\hspace{1cm}(jauge)}\\
			\square\,\phi^\mu&=-\mu_0J^\mu
		\end{array} \right.
	$$
\end{conc}

\subsubsection{Équation de Maxwell-Lorentz}
	Les quadrivecteurs ont quatre composantes, alors que $\vec{E}$ et $\vec{B}$ n'en ont chacun que trois. Il est nécessaire de considérer un tenseur d'ordre supérieur ou égal à 2. On part de l'expression de $\vec{E}$ et $\vec{B}$ avec les potentiels :

$$
	\vec{B}\nab\times\vec{A}\text{\hspace{1cm}et\hspace{1cm}}\vec{E}=-\nab\phi-\partial_t\vec{A}
$$
$$
	\begin{array}{r@{\;}l@{\;}l}
		B_x&=\partial_yA_z-\partial_zA_y=\partial_2\phi^3-\partial_3\phi^2&=-(\partial_3\phi^2-\partial_2\phi^3)\\
		B_y&=&=-(\partial_3\phi^1-\partial_1\phi^3)\\
		B_z&=&=-(\partial_1\phi^2-\partial_2\phi^1)\\	E_x&=-\partial_x\phi-\partial_tA_x=\partial_1\phi^0-\partial_0\phi^1&=-(\partial_0\phi^1-\partial_1\phi^0)\\
		E_y&=&=-(\partial_0\phi^2-\partial_2\phi^0)\\
		E_z&=&=-(\partial_0\phi^3-\partial_3\phi^0)
	\end{array}
$$

{\txt $\frac{E_x}{c}$,$\frac{E_y}{c}$,$\frac{E_z}{c}$,$B_x$,$B_y$,$B_z$ apparaissent donc comme les composantes d'un tenseur d'ordre deux asymétrique :}
$$
	\begin{array}{r@{\;}l}
		F^{\mu\nu}&=\partial^\mu\phi^\nu-\partial^\nu\phi^\mu\\
		F^{\mu\nu}&=\begin{mat}[1.7]
			0 & -\frac{E_x}{c} & -\frac{E_y}{c} & -\frac{E_z}{c}\\
			\frac{E_x}{c} & 0 & -B_z & B_y \\
			\frac{E_y}{c} & B_z & 0 & -B_x \\
			\frac{E_z}{c} & -B_y & B_x & 0 
		\end{mat}
	\end{array}
$$

Les coordonnées covariantes de ce tenseur sont :
$$
	F_{\mu\nu}=g_{\mu\alpha}g_{\nu\beta}F^{\alpha\beta}=\begin{mat}[1.7]
		0 & \frac{E_x}{c} & \frac{E_y}{c} & \frac{E_z}{c}\\
		-\frac{E_x}{c} & 0 & -B_z & B_y \\
		-\frac{E_y}{c} & B_z & 0 & -B_x \\
		-\frac{E_z}{c} & -B_y & B_x & 0
	\end{mat}
$$

On retrouve les équations de Maxwell :
$$
	\nab\cdot\vec{E}=\frac{\rho}{\epsilon_0} \text{\hspace{1.5cm}}
	\begin{array}{r@{\;}l}
		\partial_\mu F^{\mu 0}&=\partial_0F^{00}+\partial_1F^{10}+\partial_2F^{20}+\partial_3F^{30}\\
		&=0+\partial_x\left(\frac{E_x}{c}\right)+\partial_y\left(\frac{E_y}{c}\right)+\partial_z\left(\frac{E_z}{c}\right)\\
		&=\frac{1}{c}\nab\cdot\vec{E}=\frac{\rho}{c\epsilon_0}=\mu_0J^0
	\end{array}
$$
$$
	\nab\times\vec{B}=\mu_0\vec{j}+\mu_0\epsilon_0\partial_t\vec{E} \text{\hspace{1cm}}
	\begin{array}{r@{\;}l}
		\partial_\mu F^{\mu 1}&=\partial_0F^{01}+\partial_1F^{11}+\partial_2F^{21}+\partial_3F^{31}\\
		&=\partial_x\left(-\frac{E_x}{c}\right)+0+\partial_yB_z+\partial_z\left(-B_y\right)\\
		&=\left.\nab\times\vec{B}\right|_x-\frac{1}{c^2}\partial_t\left.\vec{E}\right|_x=\mu_0J_x
	\end{array}
$$

De la m\^eme manière, $\partial_\mu F^{\mu 2}$ et $\partial_\mu F^{\mu 3}$ donnent les composantes $x$ et $y$ de l'équation de Maxwell-Ampère.

Pour les équations de Maxwell avec sources, on a donc :
$$
	\boxed{\partial_\mu F^{\mu\nu}=\mu_0J^\nu}
$$
sous forme covariante. Intéressons-nous aux équations de Maxwell homogènes :
$$
	\begin{array}{r@{\;}l}
		\partial_1F_{23}+\partial_2F_{31}+\partial_3F_{12}&=\partial_x(-B_x)+\partial_y(-B_y)+\partial_z(-B_z)\\
			&= -\nab\cdot\vec{B}\\
			&= 0 \\[10pt]
		\partial_0F_{32}+\partial_3F_{20}+\partial_2F_{03}&=\frac{\partial}{c\partial t}B_x+\partial_z\left(-\frac{E_y}{c}\right)+\partial_y\left(\frac{E_z}{c}\right)\\
			&=\left.\frac{\nab\cdot\vec{E}}{c}\right|_x+\frac{1}{c}\left.\frac{\partial \vec{B}}{\partial t}\right|_x
	\end{array}
$$
M\^eme chose pour $y$ et $z$. Ces équations s'écrivent alors :
$$
	\boxed{\partial_\alpha F_{\beta\gamma} + \partial_\beta F_{\gamma\alpha} + \partial_\gamma F_{\alpha\beta} = 0}
$$

\subsubsection*{Équation de Lorentz}

$$
	m\frac{d}{dt}\frac{\vec{v}}{\sqrt{1-\frac{v^2}{c^2}}}=m\frac{d}{dt}(\gamma\vec{v})=\frac{d}{dt}(\gamma m \vec{v})=q(\vec{E}+\vec{v}\times\vec{B})
$$
et
$$
	\frac{d}{dt}(\gamma m c^2)=\bigintssss\vec{j}\cdot\vec{E}=q\vec{v}\cdot\vec{E}
$$
En utilisant le fait que
$$
	\frac{d}{dt}=\frac{d\tau}{dt}\frac{d}{d\tau}=\frac{1}{\gamma}\frac{d}{d\tau}\\
$$
on obtient
$$
	\frac{d}{d\tau}\gamma\vec{v}=q\left(\gamma c\frac{\vec{E}}{c}+\gamma\vec{v}\times\vec{B}\right)
$$
et
$$
	\frac{d}{d\tau}(\gamma mc)=q\gamma\vec{v}\cdot\frac{\vec{E}}{c}
$$
Soit
$$
	\boxed{\frac{dP^\mu}{d\tau}=qF^{\mu\nu}U_\nu}
$$
{\renewcommand*{\arraystretch}{1.2}
$$
	\begin{array}{lr@{\;}ll}
		\text{o\`u }&P^\mu&=(\gamma mc,\gamma m\vec{v})=mU^\mu&\text{ composante contravariante de la quadri-impulsion}\\
		&U^\mu&=(\gamma c,\gamma\vec{v})&\text{ composante contravariante de la quadri-vitesse}\\
		&U_\mu&=(\gamma c,-\gamma\vec{v})&\text{ composante covariante de la quadri-vitesse}
	\end{array}
$$}

\begin{proof}
$$
	\begin{array}{r@{\;}l}
		F^{0\nu}U_\nu&=F^{00}U_0+F^{01}U_1+F^{02}U_2+F^{03}U_3\\
			&=0+\left(-\frac{E_x}{c}\right)(-\gamma v_x)+\left(-\frac{E_y}{c}\right)(-\gamma v_y)+\left(-\frac{E_z}{c}\right)(-\gamma v_z)\\
			&=\gamma\vec{v}\frac{\vec{E}}{c}\\[10pt]
		F^{\nu}&=...
	\end{array}
$$
\end{proof}

\begin{conc}
On a formulé les équations de Maxwell ainsi que la force de Lorentz sous forme covariante.
\end{conc}

\subsection{Résumé}

Après avois introduit
\begin{itemize}
	\item le quadrinabla de composantes covariantes $\partial_\mu \equiv \left(\frac{\partial}{c\partial t}, \frac{\partial}{\partial x},\frac{\partial}{\partial y},\frac{\partial}{\partial z}\right)$
	\item l'opérateur d'Alembertien $\square=\nab^2-\frac{1}{c^2}\frac{\partial^2}{\partial t^2}$
	\item le quadrivecteur courant de composantes contravariantes $J^\mu=(\rho c,\vec{j})$
	\item le quadrivecteur potentiel de composantes contravariantes $\phi^\mu=\left(\frac{\phi}{c},\vec{A}\right)$
	\item le tenseur électromagnétique de composantes contravariantes $$F^{\mu\nu}=\begin{mat}[1.7]
			0 & -\frac{E_x}{c} & -\frac{E_y}{c} & -\frac{E_z}{c}\\
			\frac{E_x}{c} & 0 & -B_z & B_y \\
			\frac{E_y}{c} & B_z & 0 & -B_x \\
			\frac{E_z}{c} & -B_y & B_x & 0 
		\end{mat}$$
\end{itemize}
les équations de Maxwell prennent la forme covariante suivante :
