\chapter{Formulation relativiste}


{\small \it Note : la première partie se trouve sur une feuille distribuée par le professeur.}
\section{Rappels d'analyse tensorielle}
\section{Formulation relativiste de l'électrodynamique}
\subsection{Postulats fondamentaux}

\begin{postulat}
	Les lois de la physique, lorsqu'elles sont formulées de manière adéquate, gardent la m\^eme forme dans un référentiel $\mathcal{R}$ et dans un référentiel $\mathcal{R}'$ en translation rectiligne uniforme par rapport à $\mathcal{R}$. Il y a invariance de la forme des équations.\\
	Contribution d'Einstein : la vitesse de propagation d'un signal électromagnétique dans le vide est universelle et indépendante du référentiel.
\end{postulat}

\begin{cons}
	Ce sont les équations de Maxwell qui sont invariantes, et cela a donc amené à abandonner la mécanique classique (Newtonienne) pour la reformuler afin qu'elle soit aussi invariante sous une transformation de Lorentz.
\end{cons}

\begin{postulat}
	Invariance de la charge : en accord avec les expériences, on postule que la charge q d'une particule ne dépend pas du référentiel galiléen considéré.
\end{postulat}

\begin{cons}
	L'invariance formelle des lois de la physique est automatiquement satisfaite si on les écrit sous forme temporelle (condition suffisante), sachant que les transformations à considérer sont les transformations du groupe de Lorentz, qui recouvre les translations spatiales et temporelles, les rotations, les renversements d'axe, et les boosts de Lorentz.
$$
	x'^{\mu}=\mathcal{L}^{\mu}_{\nu}x^{\mu} \text{ avec }\mathcal{L}^{\mu}_{\nu}=\begin{pmatrix}
	\gamma & -\beta\gamma & \hspace*{0.2cm}0\hspace*{0.2cm} \\
	-\beta\gamma & \gamma & \hspace*{0.2cm}0\hspace*{0.2cm} \\
	0 & 0 & \hspace*{0.2cm}1\hspace*{0.2cm}
	\end{pmatrix}
$$
\end{cons}


\subsection{Quadrivecteur courant}
On cherche à construire un quadrivecteur à partir de $\rho$ et $\vec{j}$. Ainsi, on considère une charge $q$ dans un référentiel $\mathcal{R}$, que l'on modélise comme une distribution de charge $\rho$ dans un volume $dV=\frac{q}{\rho}$. Dans $\mathcal{R}'$ en translation rectiligne uniforme par rapport à $\mathcal{R}$, cette charge $q$ occupe $dV'$ et correspond à $\rho'$. D'après le postulat d'invariance de la charge, on a : $\rho dV=q=\rho' dV'$. Dans $\mathcal{R}$, la charge est en mouvement et est donc associée à un vecteur densité de courant :
$$
	\vec{j}=\begin{cases}
		\rho\vec{v}&\text{si }\vec{r}\in dV\\
		0&\text{sinon}
	\end{cases}
$$

Pendant un intervalle de temps $dt$ dans $\mathcal{R}$, la charge se déplace de $dx^\mu=(dt,d\vec{r})$. $dx^\mu$ étant un quadrivecteur, il en est de m\^eme pour $\rho dVdx^\mu$. On considère alors $\rho\frac{cdtdV}{c}$