\documentclass[11pt,a4paper]{report}
\usepackage[utf8]{inputenc}
\usepackage[frenchb]{babel}
\usepackage[T1]{fontenc}
\usepackage[intlimits]{amsmath}
\usepackage{amsthm}
\usepackage{amsfonts}
\usepackage{amssymb}
\usepackage[left=2cm,right=2cm,top=2cm,bottom=2cm]{geometry}


\usepackage{tabularx}
\usepackage[pdftex,colorlinks=true,pdfstartview=FitB,hyperindex=true]{hyperref}
\usepackage{bigints}

\usepackage{graphicx}
\usepackage[hang,small,bf]{caption}
\usepackage{float}

\usepackage{tikz}
\usetikzlibrary{decorations.pathmorphing,arrows}

\usepackage{mathtools}

%\usepackage{calc}
%\newlength\dlf
%\newcommand\alignedbox[2]{
%  % #1 = before alignment
%  % #2 = after alignment
%  &
%  \begingroup
%  \settowidth\dlf{$\displaystyle #1$}
%  \addtolength\dlf{\fboxsep+\fboxrule}
%  \hspace{-\dlf}
%  \boxed{#1 #2}
%  \endgroup
%}

\tikzset{
	spring/.style={decorate,
		decoration={coil,amplitude=3pt, segment length=4pt, post length=7pt, pre length=7pt}} 
	}	
	

\newcommand{\defeq}{\vcentcolon=}
\newcommand{\eqdef}{=\vcentcolon}

\newenvironment{mat}[1][1]
{\renewcommand*{\arraystretch}{#1} \begin{pmatrix}}
{\end{pmatrix}}


\theoremstyle{plain}
\newtheorem*{theorem}{Théorème}
\newtheorem*{postulat}{Postulat}

\theoremstyle{definition}
\newtheorem*{ex}{Exemple}
\newtheorem*{cons}{Conséquence}
\newtheorem*{corol}{Corollaire}
\newtheorem*{conc}{Conclusion}
\newtheorem*{app}{Application}

\theoremstyle{remark}
\newtheorem*{remark}{Remarque}
\newtheorem*{remarks}{Remarques}
\newtheorem*{rappel}{Rappel}
\newtheorem*{exo}{Exercice}


\newcommand{\txt}{%
\everymath{\textstyle}%
}

\newcommand{\para}{
\ensuremath
\mbox{\textbf{\tiny /\!/}}
}

\newcommand{\nab}{
\ensuremath
\vec{\nabla}
}

\newcommand{\bigiiintsss}{
\bigintsss\kern -6pt \bigintsss_V\kern -6pt \bigintsss \!
}

\newcommand{\dif}{
\mathrm{d}
}

\author{Christophe \textsc{Winisdoerffer}}
\title{Électrodynamique}
\date{}

\renewcommand{\thesection}{\arabic{section}}
\setcounter{secnumdepth}{3}

\everymath{\displaystyle}
\renewcommand*{\arraystretch}{1.7}

\begin{document}
 	



\thispagestyle{empty}
\newgeometry{left=3cm,right=3cm,top=8cm,bottom=2cm}
\makeatletter

\begin{flushleft}
	\Huge \@title
\end{flushleft}
\hrule height 6pt
\begin{flushright}
	Master sciences de la matière\\
	Semestre 1, ENS Lyon
\end{flushright}

\vfill
\noindent
\begin{tabularx}{\textwidth}{l X r}
	Cours de :  & & Retranscrit par : \\
	\@author & & Simon \textsc{Zugmeyer} \\
	\href{mailto:cwinisdo@ens-lyon.fr}{\tt cwinisdo@ens-lyon.fr} & &
\end{tabularx}

\makeatother
\restoregeometry

\tableofcontents
\pagebreak

\chapter{Les équations de Maxwell}


{\small \it Note : les quatre premières sections sont des polycopiés distribués par le professeur.}
\section{Introduction}
\section{Équations de Maxwell}
\section{Les invariants de Maxwell-Lorenz}
\section{Équations de Maxwell et changement de référentiel}
\subsection{Transformation de Galilée}

On considère deux référentiels $\mathcal{R}$ et $\mathcal{R}'$ en translation rectiligne uniforme l'un par rapport à l'autre, et pour lesquels on a choisi les origines spatiales et temporelles de telle façon que le système de coordonnées s'écrive :
\begin{align*}
	t'&=t\\
	\vec{r'}&=\vec{r}-\vec{v}t, \left\{ \begin{array}{r@{\;}l}
			x'&=x-v_{x}t \\
			y'&=y-v_{y}t \\
			z'&=z-v_{z}t \\
		\end{array} \right.
\end{align*}
On a alors :
\begin{align*}
	\frac{\partial}{\partial t}&=\frac{\partial t'}{\partial t} \frac{\partial}{\partial t'}
			+\frac{\partial x'}{\partial t} \frac{\partial}{\partial x'}
			+\frac{\partial y'}{\partial t} \frac{\partial}{\partial y'}
			+\frac{\partial z'}{\partial t} \frac{\partial}{\partial z'}\\
	&=\frac{\partial}{\partial t'} - v_{x} \frac{\partial}{\partial x'}
			- v_{y} \frac{\partial}{\partial y'}
			- v_{z} \frac{\partial}{\partial z'}\\
	&=\frac{\partial}{\partial t'} - \vec{v}\cdot \vec{\nabla '}\\[10pt]
	\frac{\partial}{\partial x}&=\frac{\partial t'}{\partial x} \frac{\partial}{\partial t'}
			+\frac{\partial x'}{\partial x} \frac{\partial}{\partial x'}
			+\frac{\partial y'}{\partial x} \frac{\partial}{\partial y'}
			+\frac{\partial z'}{\partial x} \frac{\partial}{\partial z'}\\
	&=\frac{\partial}{\partial x'}\\
	\frac{\partial}{\partial y}&=\frac{\partial}{\partial y'}\\
	\frac{\partial}{\partial z}&=\frac{\partial}{\partial z'}
\end{align*}
et donc 
\begin{align*}
	\nab\cdot\vec{E}(x,y,z,t)&=\vec{\nabla '}\cdot\vec{E}(x(x',y',z',t'),y(x',y',z',t'),z(x',y',z',t'),t(x',y',z',t'))\\ 
	\nab\cdot\vec{E}&=\vec{\nabla '}\cdot\vec{E}
\end{align*}

\subsubsection*{Maxwell-Faraday}
\begin{align*}
	\nab\times\vec{E}(\vec{r},t)&=-\frac{\partial \vec{B}}{t}\\
		&=\vec{\nabla '}\times\vec{E}\\
		&=-\frac{\partial \vec{B}}{t'} + (\vec{v}\cdot\vec{\nabla '})\vec{B}\\[15pt]
	\nab\times(\vec{a}\times\vec{b})&=\vec{a}(\nab\cdot\vec{b})
			-\vec{b}(\nab\cdot\vec{a})
			+(\vec{b}\cdot\nab)\vec{a}
			-(\vec{a}\cdot\nab)\vec{b}\\
	(\vec{v}\cdot\vec{\nabla '})\vec{B}&=\underbrace{\vec{v}(\vec{\nabla '}\cdot\vec{B})}_{=0}
			-\underbrace{\vec{v}(\vec{\nabla '}\cdot\vec{B})}_{=0}
			+\underbrace{(\vec{B}\cdot\vec{\nabla '})\vec{v}}_{=0}
			-\vec{\nabla '}\times(\vec{v}\times\vec{B})\\
	\vec{\nabla '}\times\vec{E}&=-\frac{\partial \vec{B}}{t'}-\vec{\nabla '}\times(\vec{v}\times\vec{B})
\end{align*}
\boxed{\vec{\nabla '}\times(\vec{E}+\vec{v}\times\vec{B})=-\frac{\partial \vec{B}}{t'}}		

\subsubsection*{Maxwell-Gauss}
\begin{align*}
{\nabla}\cdot\vec{E}=\boxed{\frac{\rho}{\epsilon_0}=\vec{\nabla '}\cdot\vec{E}}
\end{align*}

\subsubsection*{Maxwell-Ampère}
\begin{align*}
	\nab\times\vec{B}&=\mu_0\vec{j}+\mu_0\epsilon_0\frac{\partial \vec{E}}{\partial t}\\
		&=\vec{\nabla '}\times\vec{B}\\
		&=\mu_0\vec{j}+\mu_0\epsilon_0\frac{\partial \vec{E}}{\partial t'}-\mu_0\epsilon_0(\vec{v}\cdot\vec{\nabla '})\vec{E}\\[15pt]
	(\vec{v}\cdot\vec{\nabla '})\vec{E}&=\underbrace{\vec{v}(\vec{\nabla '}\cdot\vec{E})}_{=\frac{\rho}{\epsilon_0}}
			-\underbrace{\vec{v}(\vec{\nabla '}\cdot\vec{E})}_{=0}
			+\underbrace{(\vec{E}\cdot\vec{\nabla '})\vec{v}}_{=0}
			-\vec{\nabla '}\times(\vec{v}\times\vec{E})
\end{align*}
\boxed{\vec{\nabla '}\times(\vec{B}+\frac{\vec{v}\times\vec{E}}{c^2})=\mu_0(\vec{j}-\rho\vec{v})+\epsilon_0\mu_0\frac{\partial \vec{E}}{t'}}	


\subsubsection*{Conservation de la charge}
\begin{align*}
	\frac{\partial\rho}{\partial t}+\nab\cdot\vec{j}=\boxed{0=\frac{\partial\rho}{\partial t'}+\vec{\nabla '}\cdot(\vec{j}-\rho\vec{v})}
\end{align*}

\subsubsection*{Conclusion}
\begin{itemize}
	\item conservation charge $\left\{ \begin{array}{r@{\;}l}
					\rho '&=\rho\\
					\vec{j'}&=\vec{j}-\rho\vec{v}
			\end{array} \right.$
	\item Maxwell-Gauss $\vec{E '}=\vec{E}$
	\item Maxwell-flux $\vec{B '}=\vec{B}$
	\item Maxwell-Faraday $\vec{E '}=\vec{E}+\vec{v}\times\vec{B} si \vec{B'}=\vec{B}$
	\item Maxwell-Ampère $\vec{B '}=\vec{B}+\frac{\vec{v}\times\vec{E}}{c^2} si \vec{E'}=\vec{E}$
\end{itemize}

Les équations de Maxwell ne sont donc pas invariantes sous une transformation de Galilée.

\subsection{Transformation de Lorenz}
On considère deux repères $\mathcal{R}$ et $\mathcal{R}'$ en translation rectiligne uniforme l'un par rapport à l'autre, avec un choix d'origines tel que :
$\left\{ \begin{array}{r@{\;}l}
		ct'&=\gamma ct-\beta\gamma x\\
		x'&=-\beta\gamma ct+\gamma x\\
		y'&=y\\
		z'&=z
\end{array} \right.$
Dans ce cas,
\begin{align*}
	\frac{\partial}{\partial t}&=\frac{\partial t'}{\partial t} \frac{\partial}{\partial t'}
			+\frac{\partial x'}{\partial t} \frac{\partial}{\partial x'}
			+\frac{\partial y'}{\partial t} \frac{\partial}{\partial y'}
			+\frac{\partial z'}{\partial t} \frac{\partial}{\partial z'}\\
		&=\gamma\frac{\partial}{\partial t'}-\beta\gamma c\frac{\partial}{\partial x'}\\[10pt]
	\frac{\partial}{\partial x}&=-\frac{\beta\gamma}{c}\frac{\partial}{\partial t'}+\gamma\frac{\partial}{\partial x'}\\
	\frac{\partial}{\partial y}&=\frac{\partial}{\partial y'}\\
	\frac{\partial}{\partial z}&=\frac{\partial}{\partial z'}
\end{align*}

\subsubsection*{Maxwell-Faraday}
\begin{align*}
	\nab\times\vec{E}&=-\frac{\partial \vec{B}}{t}
\end{align*}
\indent$
	\left\{ \begin{array}{l}
		\partial_yE_z-\partial_zE_y=-\partial_tB_x\\
		\partial_zE_x-\partial_xE_z=-\partial_tB_y\\
		\partial_xE_y-\partial_yE_x=-\partial_tB_z\\
	\end{array} \right.\\[5pt]\indent
	\left\{
	\begin{array}{r@{\;}l}
		\partial_{y'}E_z-\partial_{z'}E_y&=-\gamma\partial_{t'}B_x+\beta\gamma c\partial_{x'}B_x\\
		\partial_{z'}E_x-\partial_{x'}(\gamma E_z)+\frac{\beta\gamma}{c}\partial_{t'}E_z&=-\gamma\partial_{t'}B_y+\beta\gamma c\partial_{x'}B_y\\
		\partial_{x'}E_y-\frac{\beta\gamma}{c}\partial_{t'}E_y-\partial_{y'}E_x&=-\gamma\partial_{t'}B_z+\beta\gamma c\partial{x'}B_z\\
	\end{array} \right.\\[10pt]\indent
	\begin{array}{r@{\;}l}
		\text{Maxwell-Flux donne : }{\nabla}\cdot\vec{B}=0&=\partial_xBx+\partial_yB_y+\partial_zB_z\\
		&=-\frac{\beta\gamma}{c}\partial_{t'}B_x+\gamma\partial_{x'}B_x+\partial_{y'}B_y+\partial_zB_z\\
		\beta\gamma c\partial_{x'}B_x&=\gamma\beta^2\partial_{t'}B_x-\beta c\partial_{y'}B_y-\beta c\partial_{z'}B_z\\[10pt]\indent
		\gamma&=\frac{1}{\sqrt{1-\beta^2}}\\
		\gamma(1-\beta^2)&=\frac{1}{\gamma}
	\end{array}\\[10pt]\indent
	\left\{
	\begin{array}{c@{-}c@{\;}l}
		\partial_{y'}(\gamma(E_z+\beta cB_y))&\partial_{z'}(\gamma(E_y-\beta cB_z))&=-\partial_{t'}B_x\\
		\partial_{z'}E_x&\partial_{x'}(\gamma(E_z+\beta cB_y))&=-\partial_{t'}(\gamma(B_y+\frac{\beta}{c}E_z))\\
		\partial_{x'}(\gamma(E_y-\beta cB_z))&\partial_{y'}E_x&=-\partial_{t'}(\gamma(B_z-\frac{\beta}{c}E_y))\\
	\end{array} \right.\\
$

En faisant le même travail pour les autres équations de Maxwell, on s'aperçoit qu'elles sont invariantes pour :\\
\indent$
\left\{ \begin{array}{r@{\;}l}
	c\rho '&=\gamma c\rho-\gamma\vec{\beta}\cdot\vec{j}\\
	\vec{j'_{\para}}&=-\vec{\beta}\gamma c\rho+\gamma\vec{j_{\para}}\\
	\vec{j'_{\perp}}&=\vec{j_{\perp}}
\end{array} \right.\\[10pt]\indent
\left\{ \begin{array}{r@{\;}l}
	\vec{E'_{\para}}&=\vec{E_{\para}}\\
	c\vec{B'_{\para}}&=c\vec{B_{\para}}\\
	\vec{E'_{\perp}}&=\gamma(\vec{E_{\perp}}+\vec{v}\times\vec{B_{\perp}})\\
	c\vec{B'_{\perp}}&=\gamma(c\vec{B_{\perp}}-\frac{\vec{v}\times\vec{E_{\perp}}}{c})
\end{array} \right.\\
$

%\[
%  \begin{array}{r@{\;}l}
%    f(x) & = a \\
%    g(x) & = ax + b \\
%    h(x) & = ax^2 + bx + c \hspace*{3em}
%      \smash{\left.\begin{array}{@{}c@{}}\\ \\ \\ \end{array}\right\}} \\
%    i(x) & = ax^3 + bx^2 + cx + d \\
%    j(x) & = ax^4 + bx^3 + cx^2 + dx + e
%  \end{array}
%\]

Remarques : 
\begin{itemize}
	\item On reconna\^it dans la transformation de $(c\rho,\vec{j})$ celle correspondant à un quadrivecteur
	\item Transformation inverse : permuter les variables primées et non primées, et changer le signe de $\beta$
	\item C'est Lorenz qui choisit l'invariance des équations de Maxwell sans les sources en 1904, et Poincaré avec les sources en 1905 en exploitant la transformation de $(c\rho,\vec{j})$
\end{itemize}

\section{Potentiels et jauges}
\subsection{Potentiels}
	Les équations de Maxwell fournissent quatre équation aux dérivées partielles couplées. Les deux équations homogènes ({\it i.e.} sans source, Maxwell-flux et Maxwell-Faraday) peuvent être identiquement résolues en introduisant le potentiel vecteur $\vec{A}(\vec{r},t)$ et le potentiel scalaire $\Phi(\vec{r},t)$ tels que :
	\begin{align*}
		\vec{B}&=\nab\times\vec{A}\\
		\vec{E}&=-\nab\Phi-\partial_t\vec{A}\\
		\text{car } \nab\cdot(\nab\times\vec{Z})&=0\\
		\text{et }\nab\times(\nab Z)&=\vec{0}
	\end{align*}
	
	Les deux autres équations deviennent alors :
	
	\begin{align*}
		\nab\vec{E}=\frac{\rho}{\epsilon_0}&=\nab\cdot(-\nab\Phi-\partial_t\vec{A})\\
		&=-\nab^2\Phi-\partial_t(\nab\cdot\vec{A})\\[10pt]
		\nab\times\vec{B}&=\mu_0\vec{j}+\mu_0\epsilon_0\partial_t\vec{E}\\
		&=\nab\times(\nab\times\vec{A})\\
		&=-\nab^2+\nab(\nab\cdot\vec{A})\\
		&=\mu_0\vec{j}+\mu_0\epsilon_0\partial_t(-\nab\phi-\partial_t\vec{A})\\
		&=\mu_0\vec{j}+\frac{1}{c^2}\nab(-\partial_t\Phi)-\frac{1}{c^2}\partial_t^2\vec{A}
	\end{align*}
	
	soit 
	\boxed{\left\{ \begin{array}{{r@{\;}l}}
		\nab^2\vec{A}-\frac{1}{c^2}\partial_t^2\vec{A}&=-\mu_0\vec{j}+\nab(\nab\cdot\vec{A}+\frac{1}{c^2}\partial_t\Phi)\\
		\nab^2\Phi-\frac{1}{c^2}\partial_t^2\Phi&=-\frac{\rho}{\epsilon_0}-\partial_t(\nab\cdot\vec{A}+\frac{1}{c^2}\partial_t\Phi) 
	\end{array}\right.}\\
	
	On est passé de quatre EDP couplées du premier ordre à deux EDP couplées du second ordre, ce qui, en soit, n'est pas utile. Cependant, on a une liberté dans le choix de $\Phi$ et $\vec{A}$, seuls $\vec{E}$ et $\vec{B}$ sont mesurables. En effet, avec la transformation :\\
\indent
	$ 
		\begin{array}{r@{\;}l}
				\vec{A'} &=\vec{A}+\nab\lambda(\vec{r},t)\\
			\Phi'&=\Phi-\partial_t\lambda(\vec{r},t)
		\end{array}\\
	$	
	Si $\lambda$ est suffisament régulier,\\
\indent
	$
		\begin{array}{r@{\;}l}
			\nab\times(\vec{A')}&=(\vec{A}+\nab\times(\nab\lambda)\\ 
			&=\vec{B}
		\end{array}\\[5pt]\indent
		\begin{array}{r@{\;}l}
			-\nab\Phi-\partial_{t'}\vec{A'}&=-\nab\Phi-\partial_t\vec{A}+\nab\partial_t\lambda-\partial_t\nab\lambda\\
			&=\vec{E}
		\end{array}
	$
	
	$(\vec{A'},\Phi')$ correspondent aux m\^emes champs $(\vec{E},\vec{B})$ que $(\vec{A},\Phi)$. Cette liberté s'appelle \emph{invariance de jauge}, et la transformation précédente \emph{transformation de jauge}. Seuls $\vec{E}$ et $\vec{B}$ sont observables ({\it i.e.} mesurables avec un détecteur local, en regardant le mouvement d'une charge), alors que $\Phi$ et $\vec{A}$ sont des intermédiaires de calcul. On se sert donc de cette liberté pour simplifier les équations.
\begin{remark}
	On ne perd rien en écrivant $\vec{E}$ et $\vec{B}$ avec $\vec{A}$ et $\Phi$. Si $\vec{E}$ et $\vec{B}$ sont solution des équations de Maxwell avec les conditions aux limites, alors on peut toujours trouver $\vec{A}$ et $\Phi$ vérifiant :
	\indent
	$
		\left\{ \begin{array}{r@{\;}l}
			\vec{B}&=\nab\times\vec{A}\\
			\vec{E}&=-\nab\Phi-\partial_t\vec{A}\\
		\end{array} \right.
	$
\end{remark}

\subsection{Jauges}
\subsubsection{Jauge de Lorenz}
	La jauge de Lorenz consiste à imposer \boxed{\nab\cdot\vec{A}+\frac{1}{c^2}\partial_t\Phi=0}
	Dans ce cas, les équations deviennent :
	\indent
	$
		\left\{ \begin{array}{c@{-}c@{\;}l}
			\nab^2\Phi&\frac{1}{c^2}\partial_t^2\Phi&=-\frac{\rho}{\epsilon_0}\\
			\nab^2\vec{A}&\frac{1}{c^2}\partial_t^2\vec{A}&=-\mu_0\vec{j}
		\end{array} \right.
	$
	\emph{i.e.} les équations sont découplées.

	\begin{theorem}
		On peut toujours trouver un couple $(\vec{A},\Phi)$ qui satisfait la jauge de Lorenz.
	\end{theorem}
	
	\begin{proof}
		Soit $(\vec{A},\Phi)$ qui ne satisfait pas la jauge de Lorenz. On cherche $(\vec{A'},\Phi')$ relié à $(\vec{A},\Phi)$ par une transformation de jauge tel qu'il satisfasse la jauge de Lorenz.\\
\indent
	$ 
		\begin{array}{r@{\;}l}
				\vec{A'} &=\vec{A}+\nab\lambda(\vec{r},t)\\
			\Phi'&=\Phi-\partial_t\lambda(\vec{r},t)
		\end{array}\\[10pt]\indent
		\begin{array}{r@{\;}l}
			\nab\cdot\vec{A'}+\frac{1}{c^2}\partial_t\Phi'&=0\\
			&=\nab\cdot(\vec{A}+\nab\lambda)+\frac{1}{c^2}\partial_t(\Phi-\partial_t\lambda)\\
			&=\nab\cdot\vec{A}+\nab^2\lambda+\frac{1}{c^2}\partial_t\Phi-\frac{\partial_t}{c^2}\lambda
		\end{array}
	$	
	\medskip
	
	Il faut résoudre $\nab^2\lambda-\frac{1}{c^2}\partial_t\lambda=-(\nab\cdot\vec{A}+\frac{1}{c^2}\partial_t\Phi)$ qui admet toujours une solution. \qedhere
	\end{proof}
	
	
	\begin{remarks} \hspace{1cm}
		\begin{itemize}
			\item La simplicité des équations sur les potentiels est telle qu'on peut aisément les résoudre à l'aide de la technique des fonctions de Green (\emph{cf.} potentiels de Liénard-Wiechert).
			\item Cette jauge est dite covariante, car on peut montrer (\emph{cf.} plus loin que si elle est satisfaite dans $\mathcal{R}$, alors elle est satisfaite dans $\mathcal{R}'$ en translation rectiligne uniforme par rapport à $\mathcal{R}$. Elle est donc particulièrement adaptée pour une description relativiste de l'électromagnétique.
			\item On remarque la relation suivante :\\
			 $-\square(\underbrace{\nab\cdot\vec{A}+\frac{1}{c^2}\partial_t\Phi}_{\text{jauge}})=\mu_0(\underbrace{\partial_t\rho+\nab\cdot\vec{j}}_{\text{conservation  charge}})$
			\hspace{1cm} où
			 $\square=\nab^2-\frac{1}{c^2}\partial_t^2$
		\end{itemize}
	\end{remarks}
	
	\subsubsection{Jauge de Coulomb}
	
	La jauge de Coulomb s'écrit \boxed{\nab\cdot\vec{A}=0}
	Les équations deviennent :\\
	\indent
	$\left\{ \begin{array}{l}
		\nab^2\Phi=-\frac{\rho}{\epsilon_0} \text{ \hspace{1cm} dont la solution est } \Phi(\vec{r},t)=\frac{1}{4\pi\epsilon_0}\int_Vd^3\vec{r'}\frac{\rho(\vec{r},t)}{\parallel\vec{r}-\vec{r'}\parallel}\\
		\nab^2-\frac{1}{c^2}\partial_t^2\vec{A}=-\mu_0\vec{j}+\nab(\frac{1}{c^2}\partial_t\Phi)	
	\end{array} \right.
	$
	
	\begin{remarks}
		\hspace{1pt}
		\begin{itemize}
			\item On sait que les signaux électromagnétiques se propagent dans le vide à la vitesse $c$, alors que $\Phi$ se "propage" instantanément. Il est à noter, à nouveau, que seuls $\vec{E}$ et $\vec{B}$ sont observables.
			\item Cette jauge n'est pas covariante, car la variable temporelle n'intervient pas.
		\end{itemize}
	\end{remarks}
	
	\begin{theorem}
		On peut toujours trouver un tel $\vec{A}$ dans un référentiel $\mathcal{R}$.
	\end{theorem}
	
	\begin{proof}
		Soit $(\vec{A},\Phi)$ qui ne satisfait pas la jauge de Coulomb. On cherche $(\vec{A'},\Phi')$ relié à $(\vec{A},\Phi)$ par une transformation de jauge tel que :\\
\indent
	$ 
		\begin{array}{r@{\;}l}
				\vec{A'} &=\vec{A}+\nab\lambda(\vec{r},t)\\
			\Phi'&=\Phi-\partial_t\lambda(\vec{r},t)
		\end{array}\\[10pt]\indent
		\nab\cdot\vec{A}=0=\nab\cdot\vec{A}+\nab^2\lambda
	$

	Il suffit de résoudre l'équation de Poisson $\nab^2\lambda=-\nab\cdot\vec{A}$ \qedhere
	\end{proof}
	
	
	\begin{remarks}\hspace{1pt}
		\begin{itemize}
			\item Cette jauge s'appelle aussi jauge électrostatique, car $\Phi$ vérifie le même type d'équation.
			\item Cette jauge s'appelle aussi jauge transverse :\\
			\indent $\begin{array}{r@{\;}l}
					\vec{j}=\vec{j_p}+\vec{j_t} \text{ avec }&\vec{j_p} \text{ longitudinal tel que } \nab\times\vec{j_p}=\vec{0}\\
					&\vec{j_t} \text{ transverse tel que } \nab\times\vec{j_t}=\vec{0}
			\end{array}\\[10pt]\indent
			\nab^2\vec{A}-\frac{1}{c^2}\partial_t^2\vec{A}=-\mu_0\vec{j_t}\\
			\text{\hspace{3cm} avec }\vec{j_t}=\frac{1}{4\pi}\nab\times\nab\times\int\frac{\vec{j}(\vec{r'},t}{\parallel\vec{r}-\vec{r'}\parallel}d^3\vec{r'}
			$
			\item Cette jauge s'appelle aussi jauge de radiation, car si on a un ensemble de charges dans un volume fini de l'espace, alors on peut montrer que le champ $\vec{E}$, $\vec{B}$ contient deux composantes, une en $\frac{1}{r}$ et l'autre en $\frac{1}{R^2}$ lorsque $R\rightarrow+\infty$. Si les charges ne sont pas accélérées, \emph{i.e.} c'est un problème de statique, alors seul le terme en $\frac{1}{r^2}$ subsiste. Le terme en $\frac{1}{R}$ est dit de radiation. Dans cette jauge, il provient exclusivement de l'équation sur $\vec{A}$ (car sur $\Phi$, c'est une équation de Poissson). C'est une jauge particulièrement bien adaptée pour la quantification du champ électromagnétique.
			\item Comme précédemment, on a :\\
					 $-\square(\underbrace{\nab\cdot\vec{A}}_{\text{jauge}})=\mu_0(\underbrace{\partial_t\rho+\nab\cdot\vec{j}}_{\text{conservation  charge}})$
			\hspace{1cm} où
			 $\square=\nab^2-\frac{1}{c^2}\partial_t^2$
		\end{itemize}
	\end{remarks}
\chapter{Formulation relativiste}


{\small \it Note : la première partie se trouve sur une feuille distribuée par le professeur.}
\section{Rappels d'analyse tensorielle}
\section{Formulation relativiste de l'électrodynamique}
\subsection{Postulats fondamentaux}

\begin{postulat}
	Les lois de la physique, lorsqu'elles sont formulées de manière adéquate, gardent la m\^eme forme dans un référentiel $\mathcal{R}$ et dans un référentiel $\mathcal{R}'$ en translation rectiligne uniforme par rapport à $\mathcal{R}$. Il y a invariance de la forme des équations.\\
	Contribution d'Einstein : la vitesse de propagation d'un signal électromagnétique dans le vide est universelle et indépendante du référentiel.
\end{postulat}

\begin{cons}
	Ce sont les équations de Maxwell qui sont invariantes, et cela a donc amené à abandonner la mécanique classique (Newtonienne) pour la reformuler afin qu'elle soit aussi invariante sous une transformation de Lorentz.
\end{cons}

\begin{postulat}
	Invariance de la charge : en accord avec les expériences, on postule que la charge q d'une particule ne dépend pas du référentiel galiléen considéré.
\end{postulat}

\begin{cons}
	L'invariance formelle des lois de la physique est automatiquement satisfaite si on les écrit sous forme temporelle (condition suffisante), sachant que les transformations à considérer sont les transformations du groupe de Lorentz, qui recouvre les translations spatiales et temporelles, les rotations, les renversements d'axe, et les boosts de Lorentz.
$$
	x'^{\mu}=\mathcal{L}^{\mu}_{\nu}x^{\mu} \text{ avec }\mathcal{L}^{\mu}_{\nu}=\begin{mat}
	\gamma & -\beta\gamma & \hspace*{0.2cm}0\hspace*{0.2cm} \\
	-\beta\gamma & \gamma & \hspace*{0.2cm}0\hspace*{0.2cm} \\
	0 & 0 & \hspace*{0.2cm}1\hspace*{0.2cm}
	\end{mat}
$$
\end{cons}


\subsection{Quadrivecteur courant}
{\txt
On cherche à construire un quadrivecteur à partir de $\rho$ et $\vec{j}$. Ainsi, on considère une charge $q$ dans un référentiel $\mathcal{R}$, que l'on modélise comme une distribution de charge $\rho$ dans un volume $dV=\frac{q}{\rho}$. Dans $\mathcal{R}'$ en translation rectiligne uniforme par rapport à $\mathcal{R}$, cette charge $q$ occupe $dV'$ et correspond à $\rho'$. D'après le postulat d'invariance de la charge, on a : $\rho dV=q=\rho' dV'$. Dans $\mathcal{R}$, la charge est en mouvement et est donc associée à un vecteur densité de courant :}
$$
	\vec{j}=\begin{cases}
		\rho\vec{v}&\text{si }\vec{r}\in dV\\
		0&\text{sinon}
	\end{cases}
$$

{\txt
Pendant un intervalle de temps $dt$ dans $\mathcal{R}$, la charge se déplace de $dx^\mu=(dt,d\vec{r})$. $dx^\mu$ étant un quadrivecteur, il en est de m\^eme pour $\rho dVdx^\mu$. On considère alors $\rho\frac{cdtdV}{c}\frac{dx^\mu}{dt}$}.

\begin{postulat}
	$cdtdV$ est un invariant.
\end{postulat}

\begin{proof}
	$cdtdV$ étant l'élément d'intégration dans l'espace de Minkovski, on considère la matrice jacobienne.
\begin{rappel}
	Soit $\varphi:(x,y,z)\longrightarrow \vec{\varphi}(x,y,z)=\begin{mat}
		u(x,y,z)\\
		v\\
		w
	\end{mat}$
	
	$$
		\text{Jac}_\varphi=\frac{D(u,v,w)}{D(x,y,z)}=\begin{mat}
			\partial_xu & \partial_yu & \partial_zu\\
			\partial_xv & \partial_yv & \partial_zv\\
			\partial_xw & \partial_yw & \partial_zw
		\end{mat}\\
	$$
	$$
		J_\varphi=\det(\text{Jac}_\varphi)
	$$
	$$
		\text{alors }\bigintssss_{\varphi(x)}f(u_1,...u_n)\,du_1...du_n=\bigintssss_Xf\circ\varphi(x_1,...x_n)|J_\varphi|\,dx_1...dx_n
	$$
\end{rappel}
Pour un boost de Lorentz, on a :
$$
	\begin{mat}
		ct' \\ x' \\ t' \\ z'
	\end{mat}
	=
	\begin{mat}
		\gamma & -\beta\gamma & 0 & 0\\
		-\beta\gamma & \gamma & 0 & 0\\
		0 & 0 & 1 & 0\\
		0 & 0 & 0 & 1
	\end{mat}
	\begin{mat}
		ct \\ x \\ y \\ z
	\end{mat}
	\text{ et par conséquent }J_{boost}=1
$$
$$
	\begin{array}{r@{\;}l}
		\text{d'où\hspace*{0.5cm}} cdtdxdydz&=cdt'dx'dy'dz'\\
		cdtdV&=cdt'dV'
	\end{array}
$$
\end{proof}

\begin{remark}
{\txt Attention aux démonstrations avec $dt'=\gamma dt$ et $dx'=\frac{1}{\gamma}dx$}
\end{remark}

\begin{corol}
	{\txt $cdtdxdydz$ est invariant et donc $\rho\frac{dx^\mu}{dt}$ est un quadrivecteur.}\\
	On pose $J^\mu=\rho\frac{dx^\mu}{dt}$ le quadrivecteur courant (composantes contravariantes).\\
	$J^\mu=(\rho c,\vec{j})=(\rho c,\rho \vec{v})=\rho_0(\gamma c,\gamma\vec{v})$ où $\rho_0$ est la densité de charges dans le référentiel propre de la particule.
\end{corol}

\subsection{Formulation covariante de l'électromagnétisme}
\subsubsection{Retour sur un opérateur}

On a vu que $\nabla$ est un quadrivecteur dont les composantes covariantes sont $\partial_a=\frac{\partial}{\partial{x^a}}$ soit :
$$ 
	\nabla=\left(\frac{1}{c}\frac{\partial}{\partial t},\frac{\partial}{\partial x},\frac{\partial}{\partial y},\frac{\partial}{\partial z}\right)
$$
et ses composantes contravariantes s'écrivent : $\left(\frac{1}{c}\frac{\partial}{\partial t},-\frac{\partial}{\partial x},-\frac{\partial}{\partial y},-\frac{\partial}{\partial z}\right)$\\
$\partial^a\partial_a$ est un tenseur d'ordre 0 obtenu par contraction de deux tenseurs de rang 1 et est donc invariant.
$$
	\begin{array}{r@{\;}l}
			\partial^a\partial_a&=\partial^0\partial_0+\partial^1\partial_1+\partial^2\partial_2+\partial^3\partial_3\\
		&=\frac{1}{c^2}\frac{\partial^2}{\partial t^2}-\frac{\partial^2}{\partial x^2}-\frac{\partial^2}{\partial y^2}-\frac{\partial^2}{\partial z^2} = -\square
	\end{array}	
$$

\subsubsection{Quadrivecteur potentiel}
En jauge de Lorenz ($\nab\cdot\vec{A}+\frac{1}{c^2}\partial_t\phi=0$), les équations satisfaites par les potentiels sont :
$$
	\left\{ \begin{array}{r@{\;}l}
		\nab^2\phi - \frac{1}{c^2}\partial_t^2\phi&=-\frac{\rho}{\epsilon_0}\\
		\nab^2\vec{A}-\frac{1}{c^2}\partial_t^2\vec{A}&=-\mu_0\vec{j}\\
		\square\,\frac{\phi}{c}&=-\frac{\rho}{c\epsilon_0}=-\mu_0\rho c\\
		\square\,\vec{A}&=-\mu_0\vec{j}
	\end{array} \right.
$$
{\txt On est amené à considérer $(\frac{\phi}{c},\vec{A})$ comme les composantes contravariantes d'un quadrivecteur, le quadripotentiel.}
$$
	\square\,\phi^\mu=-\mu_0J^\mu
	\text{ avec } \phi^\mu=(\frac{\phi}{c},\vec{A})
$$
En ce qui concerne la jauge de Lorenz,
$$
	\begin{array}{r@{\;}l}
		0&=\nab\cdot\vec{A}+\frac{1}{c^2}\partial_t\phi=\partial_\mu\phi^\mu\\
		&=\partial_0\phi^0+\partial_1\phi^1+\partial_2\phi^2+\partial_3\phi^3\\
		&=\frac{1}{c}\partial_t\left(\frac{\phi}{c}\right)+\underbrace{\partial_xA_x+\partial_yA_y+\partial_zA_z}_{\nab\cdot\vec{A}}
	\end{array}
$$
La jauge est covariante.

\begin{conc}
	On a été capable de reformuler l'électromagnétique sous forme covariante :
	$$
		\left\{ \begin{array}{r@{\;}l}
			\partial_\mu\phi^\mu&=0\text{\hspace{1cm}(jauge)}\\
			\square\,\phi^\mu&=-\mu_0J^\mu
		\end{array} \right.
	$$
\end{conc}

\subsubsection{Équation de Maxwell-Lorentz}
	Les quadrivecteurs ont quatre composantes, alors que $\vec{E}$ et $\vec{B}$ n'en ont chacun que trois. Il est nécessaire de considérer un tenseur d'ordre supérieur ou égal à 2. On part de l'expression de $\vec{E}$ et $\vec{B}$ avec les potentiels :

$$
	\vec{B}\nab\times\vec{A}\text{\hspace{1cm}et\hspace{1cm}}\vec{E}=-\nab\phi-\partial_t\vec{A}
$$
$$
	\begin{array}{r@{\;}l@{\;}l}
		B_x&=\partial_yA_z-\partial_zA_y=\partial_2\phi^3-\partial_3\phi^2&=-(\partial_3\phi^2-\partial_2\phi^3)\\
		B_y&=&=-(\partial_3\phi^1-\partial_1\phi^3)\\
		B_z&=&=-(\partial_1\phi^2-\partial_2\phi^1)\\	E_x&=-\partial_x\phi-\partial_tA_x=\partial_1\phi^0-\partial_0\phi^1&=-(\partial_0\phi^1-\partial_1\phi^0)\\
		E_y&=&=-(\partial_0\phi^2-\partial_2\phi^0)\\
		E_z&=&=-(\partial_0\phi^3-\partial_3\phi^0)
	\end{array}
$$

{\txt $\frac{E_x}{c}$,$\frac{E_y}{c}$,$\frac{E_z}{c}$,$B_x$,$B_y$,$B_z$ apparaissent donc comme les composantes d'un tenseur d'ordre deux asymétrique :}
$$
	\begin{array}{r@{\;}l}
		F^{\mu\nu}&=\partial^\mu\phi^\nu-\partial^\nu\phi^\mu\\
		F^{\mu\nu}&=\begin{mat}[1.7]
			0 & -\frac{E_x}{c} & -\frac{E_y}{c} & -\frac{E_z}{c}\\
			\frac{E_x}{c} & 0 & -B_z & B_y \\
			\frac{E_y}{c} & B_z & 0 & -B_x \\
			\frac{E_z}{c} & -B_y & B_x & 0 
		\end{mat}
	\end{array}
$$

Les coordonnées covariantes de ce tenseur sont :
$$
	F_{\mu\nu}=g_{\mu\alpha}g_{\nu\beta}F^{\alpha\beta}=\begin{mat}[1.7]
		0 & \frac{E_x}{c} & \frac{E_y}{c} & \frac{E_z}{c}\\
		-\frac{E_x}{c} & 0 & -B_z & B_y \\
		-\frac{E_y}{c} & B_z & 0 & -B_x \\
		-\frac{E_z}{c} & -B_y & B_x & 0
	\end{mat}
$$

On retrouve les équations de Maxwell :
$$
	\nab\cdot\vec{E}=\frac{\rho}{\epsilon_0} \text{\hspace{1.5cm}}
	\begin{array}{r@{\;}l}
		\partial_\mu F^{\mu 0}&=\partial_0F^{00}+\partial_1F^{10}+\partial_2F^{20}+\partial_3F^{30}\\
		&=0+\partial_x\left(\frac{E_x}{c}\right)+\partial_y\left(\frac{E_y}{c}\right)+\partial_z\left(\frac{E_z}{c}\right)\\
		&=\frac{1}{c}\nab\cdot\vec{E}=\frac{\rho}{c\epsilon_0}=\mu_0J^0
	\end{array}
$$
$$
	\nab\times\vec{B}=\mu_0\vec{j}+\mu_0\epsilon_0\partial_t\vec{E} \text{\hspace{1cm}}
	\begin{array}{r@{\;}l}
		\partial_\mu F^{\mu 1}&=\partial_0F^{01}+\partial_1F^{11}+\partial_2F^{21}+\partial_3F^{31}\\
		&=\partial_x\left(-\frac{E_x}{c}\right)+0+\partial_yB_z+\partial_z\left(-B_y\right)\\
		&=\left.\nab\times\vec{B}\right|_x-\frac{1}{c^2}\partial_t\left.\vec{E}\right|_x=\mu_0J_x
	\end{array}
$$

De la m\^eme manière, $\partial_\mu F^{\mu 2}$ et $\partial_\mu F^{\mu 3}$ donnent les composantes $x$ et $y$ de l'équation de Maxwell-Ampère.

Pour les équations de Maxwell avec sources, on a donc :
$$
	\boxed{\partial_\mu F^{\mu\nu}=\mu_0J^\nu}
$$
sous forme covariante. Intéressons-nous aux équations de Maxwell homogènes :
$$
	\begin{array}{r@{\;}l}
		\partial_1F_{23}+\partial_2F_{31}+\partial_3F_{12}&=\partial_x(-B_x)+\partial_y(-B_y)+\partial_z(-B_z)\\
			&= -\nab\cdot\vec{B}\\
			&= 0 \\[10pt]
		\partial_0F_{32}+\partial_3F_{20}+\partial_2F_{03}&=\frac{\partial}{c\partial t}B_x+\partial_z\left(-\frac{E_y}{c}\right)+\partial_y\left(\frac{E_z}{c}\right)\\
			&=\left.\frac{\nab\cdot\vec{E}}{c}\right|_x+\frac{1}{c}\left.\frac{\partial \vec{B}}{\partial t}\right|_x
	\end{array}
$$
M\^eme chose pour $y$ et $z$. Ces équations s'écrivent alors :
$$
	\boxed{\partial_\alpha F_{\beta\gamma} + \partial_\beta F_{\gamma\alpha} + \partial_\gamma F_{\alpha\beta} = 0}
$$

\subsubsection*{Équation de Lorentz}

$$
	\begin{array}{r@{\;}l}
		m\frac{d}{dt}\frac{\vec{v}}{\sqrt{1-\frac{v^2}{c^2}}}&=q(\vec{E}+\vec{v}\times\vec{B})\\
		m\frac{d}{dt}\gamma\vec{v}&=q(\vec{E}+\vec{v}\times\vec{B})\\
		\frac{d}{dt}(\gamma m \vec{v})&=q(\vec{E}+\vec{v}\times\vec{B})\\
		\frac{d}{dt}(\gamma m c^2)&=
	\end{array}
$$
\chapter{Formulation lagrangienne de l'électrodynamique}

\section{Mécanique du point et approche lagrangienne}
{\small \it Note : la première sous-partie se trouve sur une feuille distribuée par le professeur.}
\subsection{Rappel de mécanique analytique classique}
\subsection{Mécanique analytique et relativité}
\subsubsection{Approche lagrangienne}

On a vu (au moins dans le cas de l'électrodynamique) qu'on peut écrire le principe fondamental de la dynamique sous forme covariante :
$$
	\frac{\dif P^\mu}{\dif \tau}=K^\mu
$$
où $K^\mu$ est la quadri-force de Minkowski qui s'écrit en électrodynamique $qF^{\mu\nu}U_{\mu\nu}$. Le but de cette section est de donner une formulation lagrangienne covariante équivalente, \emph{ie.} exhiber un Lagrangien tel que les équations d'Euler-Lagrange soient le principe fondamental de la dynamique.

\emph{Difficultés }: en mécanique analytique classique, on introduit l'action :
$$
	S=\bigintssss_{t_1}^{t_2}\dif (\{q_i\},\{\dot{q_i}\},\{t\})\,\dif t
$$
et $t$ joue un rôle particulier incompatible avec la relativité. En effet, si $u^\mu=(\gamma c,\gamma \vec{v})$ désigne la quadri-vitesse, $u^\mu u_\mu = c^2$ et les composantes de la quadri-vitesse ne sont pas indépendantes.
	
{\txt \emph{Retour sur la notion de trajectoire }: c'est une courbe paramétrée par un unique paramètre pour lequel la seule contrainte est d'en faire une fonction monotone. En mécanique classique, le choix naturel est $t$. En mécanique relativiste, on introduit un paramètre $\theta$ tel que l'évolution de $\theta$ soit monotone le long de la trajectoire (entendue comme la ligne d'univers suivie par la particule dans l'espace de Minkowski). On impose que $\theta$ soit un invariant de Lorentz. On pourrait prendre le temps propre $\tau$ mais rien ne nous y oblige, et un tel choix ne permettrait pas de décrire une particule sans masse. À partir de là, on introduit un Lagrangien $\Lambda(x^\mu,\frac{\dif x^\mu}{\dif\theta})$ et une action $\mathcal{S}$ :}
$$
	\mathcal{S}\defeq\bigintsss_{\text{\tiny évènement 1}}^{\text{\tiny évènement 2}} \Lambda(x^\mu,\frac{\dif x^\mu}{\dif\theta})\,\dif\theta
$$
{\txt On pose $\dot{x}=\frac{\dif x}{\dif\theta}$. Il est nécessaire que $S$ soit un invariant de Lorentz pour que la notion d'extrémalisation de $\mathcal{S}$ reste valable quel que soit le référentiel considéré. $\Lambda(x^\mu,\dot{x}^\mu)$ est donc un invariant de Lorentz. Le but est alors de l'expliciter sous forme covariante pour que les équations d'Euler-Lagrange soient équivalentes au principe fondamental de la dynamique :}
$$
	\frac{\dif}{\dif\theta}\left(\frac{\partial \Lambda}{\partial\dot{x}^\mu}\right)-\frac{\partial\Lambda}{\partial x^\mu}=0
$$ 
\begin{remark}
	$\Lambda(x^\mu,\dot{x}^\mu)$ recouvre le cas où le Lagrangien dépend de $t$ explicitement car $x^0=t$.
\end{remark}

{\txt Un choix naturel pour $\theta$ est de considérer le temps propre $\tau$, mais dans ce cas, $\frac{\dif x^\mu}{\dif\tau}\frac{\dif x_\mu}{\dif\tau}=u^\mu u_\mu=c^2$. Pour prendre en compte cette contrainte, on introduit le multiplicateur de Lagrange $\frac{\lambda}{2}$ et on construit :}
$$
	\tilde{\Lambda}=\Lambda(x^\mu,\dot{x}^\mu)+\frac{\lambda}{2}(u^\mu u_\mu-c^2)
$$
et on extrémalise 
$$
	\mathcal{S}=\bigintssss_{\tau_1}^{\tau_2} \tilde{\Lambda}(x^\mu,\dot{x}^\mu)\,\dif\tau
$$
sous contrainte. Les équations de Lagrange s'écrivent alors :
$$
	\left\{ \begin{array}{l}
		\frac{\partial\Lambda}{\partial x^\mu}-\frac{\dif}{\dif\tau}\left(\frac{\partial \Lambda}{\partial\dot{x}^\mu}\right)-\frac{\dif}{\dif\tau}(\lambda u_\mu)=0	\\
		u^\mu u_\mu-c^2=0
	\end{array} \right.
$$
$$
	u^\mu \frac{\dif}{\dif\tau}(\lambda u_\mu)=u^\mu\left(\frac{\partial\Lambda}{\partial x^\mu}-\frac{\dif}{\dif\tau}\left(\frac{\partial \Lambda}{\partial u^\mu}\right)\right)
$$
$$
	\begin{array}{lr@{\;}l}
		\text{D'où}&\frac{\dif \lambda}{\dif\tau}&=u^\mu\left(\frac{\partial\Lambda}{\partial x^\mu}-\frac{\dif}{\dif\tau}\left(\frac{\partial \Lambda}{\partial u^\mu}\right)\right)\\
		\text{Donc}&\lambda &=\bigintsss\dif\tau\,\left( u^\mu \frac{\partial\Lambda}{\partial x^\mu}-u^\mu \frac{\dif}{\dif\tau}\left(\frac{\partial \Lambda}{\partial u^\mu}\right)\right)\\
		&&=\bigintsss\dif\tau\,\left(u^\mu \frac{\partial \Lambda}{\partial u^\mu}+ \frac{\dif u^\mu}{\dif\tau}\frac{\partial\Lambda}{\partial u^\mu}-\frac{\dif u^\mu}{\dif\tau}\frac{\partial\Lambda}{\partial u^\mu}-u^\mu \frac{\dif}{\dif\tau}\left(\frac{\partial \Lambda}{\partial u^\mu}\right) \right)\\
		&&=\bigintsss\dif\tau\,\frac{\dif\Lambda(x^\mu ,u^\mu)}{\dif\tau} - \bigintsss\dif\tau\,\frac{\dif}{\dif\tau}\left(u^\mu\frac{\partial \Lambda}{\partial u^\mu} \right) \\
		&&=\Lambda-u^\nu\frac{\partial \Lambda}{\partial u^\nu}
	\end{array}
$$
Les équations d'Euler-Lagrange deviennent :  
$$
	\frac{\partial\Lambda}{\partial x^\mu}-\frac{\dif}{\dif\tau}\left(\frac{\partial \Lambda}{\partial u^\mu}\right)-\frac{\dif}{\dif\tau}\left(\left(\Lambda-u^\mu\frac{\partial\Lambda}{\partial u^\mu}\right) u^\mu\right)=0
$$

Soit $\theta$ un paramètre affine convenable. Alors, pour tout $\sigma \in \mathcal{C}^1$ tel que $\frac{\dif\sigma}{\dif\theta}=\lambda>0$, on impose que :
$$
	\mathcal{S}=\bigintsss_{\theta_1}^{\theta_2}\dif\theta\,\Lambda\!\left(x^\mu,\frac{\dif x^\mu}{\dif\theta}\right) \text{ soit égale à } \bigintsss_{\sigma(\theta_1)}^{\sigma(\theta_2)}\dif\sigma\,\Lambda\!\left(x^\mu,\frac{\dif x^\mu}{\dif\sigma}\right) 
$$
$\mathcal{S}$ est dite \emph{invariante sous une paramétrisation affine}. On a alors :
$$
	\begin{array}{r@{\;}l}
		\mathcal{S}&=\bigintsss_{\theta_1}^{\theta_2}\dif\theta\,\Lambda\!\left(x^\mu,\frac{\dif x^\mu}{\dif\theta}\right)\\
			&=\bigintsss_{\sigma(\theta_1)}^{\sigma(\theta_2)}\dif\sigma\,\Lambda\!\left(x^\mu,\frac{\dif x^\mu}{\dif\sigma}\frac{1}{\frac{\dif\sigma}{\dif\theta}}\right)\\
			&=\bigintsss_{\sigma_1}^{\sigma_2}\dif\sigma\,\Lambda\!\left(x^\mu,\frac{\dif x^\mu}{\dif\sigma}\right)
	\end{array}
$$

{\txt $ \Lambda(x^\mu,\dot{x}^\mu)$ est une fonction homogène du premier degré par rapport à la vitesse, \emph{i.e.} pour tout $\lambda>0, \Lambda(x^\mu,\lambda\dot{x}^\mu)=\lambda\Lambda(x^\mu,\dot{x}^\mu)$.
Le théorème d'Euler assure que $\Lambda(x^\mu,\dot{x}^\mu)=\dot{x}^\mu\frac{\partial\Lambda}{\partial\dot{x}^\mu}$}

Regardons :
$$
	\begin{array}{l@{\;}l}
		\left(\frac{\dif}{\dif\theta}\left(\frac{\partial\Lambda}{\partial\dot{x}^\mu}\right)-\frac{\partial\Lambda}{\partial x^\mu}\right)\dot{x}^\mu&= \frac{\dif}{\dif \theta}\left(\frac{\partial\Lambda}{\partial\dot{x}^\mu}\dot{x}^\mu\right)-\frac{\partial \Lambda}{\partial \dot{x}^\mu}\frac{\dif \dot{x}^\mu}{\dif\theta}-\frac{\partial \Lambda}{\partial x^\mu}\dot{x}^\mu\\
			&=\frac{\dif}{\dif\theta}\left(\frac{\partial\Lambda}{\partial\dot{x}^\mu}\dot{x}^\mu\right)-\frac{\dif\Lambda}{\dif \theta}\\
			&=\frac{\dif}{\dif\theta}\left(\dot{x}^\mu\frac{\partial \Lambda}{\partial \dot{x}^\mu}-\Lambda\right)\\
			&=0
	\end{array}
$$

\begin{conc}
	Modulo la remarque de Dirac (\og  $u^\mu u_\mu-c^2=0$ est une équation faible \fg (\emph{NDR : définition d'une équation faible introuvable sur internet}) \emph{i.e.} on peut faire comme si elle n'existait pas pour les calculs, et s'en souvenir à la fin pour normaliser les résultats), on retrouve les équations d'Euler-Lagrange :
$$
	\frac{\dif}{\dif \theta}\left(\frac{\partial\Lambda}{\partial \dot{x}^\mu}\right)-\frac{\partial\Lambda}{\partial x^\mu}=0
$$
\end{conc}

Il reste à expliciter $\Lambda(x^\mu,\dot{x}^\mu)$. On s'appuit sur les faits suivants :
\begin{itemize}
	\item on doit retrouver le principe fondamental de la dynamique ;
	\item $\Lambda$ doit tendre vers sa formulation non relativiste dans la limite non relativiste ;
	\item $\Lambda$ doit s'écrire sous forme covariante (pour assurer l'invariance de Lorentz). 
\end{itemize}

{\txt Dans le cas d'une particule libre, on sait expliciter le Lagrangien lorsque l'on distingue les conditions temporelle et spatiales, et $\mathcal{L}_{NR}=\frac{1}{2}m\vec{v}^2+\mathrm{cte}$.}
$$
	\begin{array}{r@{\;}l}
		\mathcal{S}=\bigintsss_{t_1}^{t_2} \! \dif t\,\mathcal{L}\!\left(x,t,\frac{\dif x}{\dif t}\right) &= \bigintsss_{\theta_1}^{\theta_2} \!\dif \theta\,\frac{\dif t}{\dif\theta}\mathcal{L}\!\left(x^{\alpha},\frac{\dif x^{\alpha}}{\dif \theta}\frac{\dif\theta}{\dif t}\right)\\
		&\defeq  \bigintsss_{\theta_1}^{\theta_2} \!\dif \theta\,\Lambda\!\left(x^{\mu},\frac{\dif x^{\mu}}{\dif \theta}\right)
	\end{array}
$$
\emph{i.e.} $\frac{\dif t}{\dif\theta}\mathcal{L}\!\left(x^{\alpha},\frac{\dif x^{\alpha}}{\dif \theta}\frac{\dif\theta}{\dif t}\right) =\Lambda\!\left(x^{\mu},\frac{\dif x^{\mu}}{\dif \theta}\right)$ et on choisit $\tau=\theta$, $\frac{\dif t}{\dif \tau}=\gamma$.
$\mathcal{L}$ ne peut dépendre que des vitesses, et $\mathcal{L}=-mc^2\sqrt{1-\frac{v^2}{c^2}}$ est covariant.

$$
	\begin{array}{r@{\;}l}
		\mathcal{L}&\xrightarrow{\text{non relativiste}}-mc^2+\frac{1}{2}m\vec{v}^2=\mathcal{L}_{NR}\\
		\mathcal{L}&=-mc\sqrt{c^2-\vec{v}^2}\\
			&=-\frac{mc}{\gamma}\sqrt{\gamma^2c^2-\gamma^2\vec{v}^2}\\
			&=-\frac{mc}{\gamma}\sqrt{\frac{\dif x^{\alpha}}{\dif\tau}\frac{\dif x_{\alpha}}{\dif\tau}}
	\end{array}$$

Soit finalement : $\Lambda(x^\mu,\dot{x}^\mu)=-mc\sqrt{\frac{\dif x^{\alpha}}{\dif\tau}\frac{\dif x_{\alpha}}{\dif\tau}}$ est covariant.

Retrouve-t-on le principe fondamental de la dynamique ? Les équations d'Euler-Lagrange donnent :
$$
	\begin{array}{r@{\;}l}
		0&=\frac{\dif}{\dif \tau}\left(\frac{\partial\Lambda}{\partial\dot{x}^\mu}\right)-\frac{\partial\Lambda}{\partial x^\mu}\\
			&=\frac{\dif}{\dif\tau}\frac{\partial}{\partial\dot{x}^\mu}\left(-mc\sqrt{g_{\alpha\beta}\dot{x}^{\alpha}\dot{x}^{\beta}}\right)\\
			&=-mc\frac{\dif}{\dif\tau}\left(\frac{1}{2\sqrt{g_{\alpha\beta}\dot{x}^{\alpha}\dot{x}^{\beta}}}( g_{\alpha\beta}\dot{x}^{\beta}\delta_{\alpha\mu}+g_{\alpha\beta}\dot{x}^{\alpha}\delta_{\beta\mu})\right)\\
			&=-mc\frac{\dif}{\dif\tau}\frac{1}{\sqrt{g_{\alpha\beta}\dot{x}^{\alpha}\dot{x}^{\beta}}}g_{\mu\beta}\dot{x}^{\beta}\\
			&=-mc\frac{\dif}{\dif\tau}\frac{\dot{x}_\mu}{\sqrt{g_{\alpha\beta}\dot{x}^{\alpha}\dot{x}^{\beta}}}
	\end{array}
$$
Soit directement : $\frac{\dif\dot{x^\mu}}{\dif\tau}=0$, en accord avec le principe fondamental de la dynamique $\frac{\dif P^\mu}{\dif\tau}=0$.

\begin{remarks}\hspace*{1pt}
	\begin{itemize}
		\item 
		$$
			\begin{array}{r@{\;}l}
				\dot{x}^\mu\frac{\partial\Lambda}{\partial\dot{x}^\mu}&=\frac{\dot{x}^\mu\dot{x}_\mu}{\sqrt{g_{\alpha\beta}\dot{x}^{\alpha}\dot{x}^{\beta}}}\\
					&=mc\sqrt{g_{\alpha\beta}\dot{x}^{\alpha}\dot{x}^{\beta}}\\
					&=\Lambda \text{ est homogène du premier degré par rapport à la vitesse}
			\end{array}
		$$
		\item {\txt Le point clé pour calculer les géodésiques en relativité générale consiste à dire que $\tau=\int_{\theta_1}^{\theta_2}\!\dif\theta\sqrt{g_{\alpha\beta}\dot{x}^{\alpha}\dot{x}^{\beta}}$ est stationnaire mais que $g_{\alpha\beta}$ dépend des coordonnées.} 
		\begin{exo}
			{\txt De la m\^eme manière, on montre que
			$$
				\Lambda=-\left(mc\sqrt{g_{\alpha\beta}\dot{x}^{\alpha}\dot{x}^{\beta}}+qA^\mu\frac{\dif x_\mu}{\dif\tau}\right)\text{\hspace{10pt}avec }A^\mu=\left(\frac{\phi}{c},\vec{A}\right)
			$$
			permet de retrouver le principe fondamental de la dynamique $\frac{\dif P^\mu}{\dif\tau}=qF^{\mu\nu}u_\nu$ pour une particule dans un champ électromagnétique.}
		\end{exo}
		\item {\txt En mécanique classique, le Lagrangien est défini à $\frac{\dif F(q,t)}{\dif t}$ près ; en mécanique relativiste, $\Lambda$ est défini à $\frac{\dif\Lambda}{\dif\theta}$ près.}
	\end{itemize}
\end{remarks}

\subsubsection{Approche hamiltonienne}
Tout comme en mécanique classique, on introduit 
\begin{itemize}
	\item le moment conjugué
	$$
		p_{\alpha}\defeq\frac{\partial\Gamma}{\partial\dot{x}^{\alpha}}\text{\hspace{10pt}où 	}\dot{x}^{\alpha}=\frac{\dif x^{\alpha}}{\dif\theta}
	$$
	\item le hamiltonien
	$$
		\mathcal{H}(x^\mu,p^\mu)\defeq p^{\alpha}\dot{x}_{\alpha}-\Lambda
	$$
\end{itemize}
et on retrouve l'équation de Hamilton :
$$
	\left\{ \begin{array}{r@{\;}l}
		\frac{\partial\mathcal{H}}{\partial x^\mu}&=-\dot{p}^\mu\\
		\frac{\partial\mathcal{H}}{\partial p^\mu}&=\dot{x}^\mu\\
	\end{array} \right.
$$


\section{Champ de mécanique analytique}
On a été capable de retrouver la formulation lagrangienne autant en mécanique classique qu'en mécanique relativiste pour un ensemble de particules éventuellement placé dans un champ. Mais on a également été capable d'expliciter les lois de conservation (pour l'impulsion, l'énergie) lorsque le système considéré est l'ensemble constitué de la particule et du champ. Peut-on formuler sous forme lagrangienne les lois de la dynamique lorsque l'on n'a pas un ensemble dénombrable de degrés de liberté ?

\subsection{Champ en mécanique classique analytique}
\subsubsection{Du discret au continu}

Soit une cha\^ine d'oscillateurs harmoniques unidimensionnels :
\begin{figure}[H]
\centering
\begin{tikzpicture}[scale=1]
	\begin{scope}
	\clip (1.2,1) rectangle (10.8,-1);		
		\foreach \x in {0,2,...,12}
			{
			\draw[spring,red,-*] (\x,0) -- +(2,0);
			\node at (\x+1,10 pt) {$k$};
			\node at ((\x +1.95,-8 pt) {$m$};
			}
	\end{scope}
			
	\draw [->] (1,-1) -- (11,-1) node [right] {$z$};

	\draw (2,-1cm-1pt) -- (2,-1cm+1pt) node [below] {$z_{i-2}$};
	\draw (4,-1cm-1pt) -- (4,-1cm+1pt) node [below] {$z_{i-1}$};
	\draw (6,-1cm-1pt) -- (6,-1cm+1pt) node [below] {$z_{i}$};
	\draw (8,-1cm-1pt) -- (8,-1cm+1pt) node [below] {$z_{i+1}$};
	\draw (10,-1cm-1pt) -- (10,-1cm+1pt) node [below] {$z_{i+2}$};
\end{tikzpicture}
\caption*{Chaîne d'oscillateurs harmoniques}
\end{figure}

Pour tout $i$, on définit $z_{i+1}^0-z_i^0\eqdef b$ comme la longueur à vide des ressorts.
$$
	\begin{array}{r@{\;}l}
		U&=\sum \frac{1}{2}k(u_{i+1}-u_i)^2 \text{ où } u_i=z_i-z_i^0\\
		T&=\sum \frac{1}{2}m\dot{u}_i\\
		L&=T-U=\sum b\left(\frac{1}{2}\frac{m}{b}\dot{u}_i^2-\frac{1}{2}kb\left(\frac{u_{i+1}-u_i}{b}\right)^2\right)=\sum L_i
	\end{array}
$$
Les équations d'Euler-Lagrange donnent :
$$
	\begin{array}{c}
		\frac{\dif }{\dif t}\frac{\partial L}{\partial \dot{u}_i}-\frac{\partial L}{\partial u_i}=0\\
		m\ddot{u}_i+kb\left(\frac{u_i-u_{i+1}}{b^2}+\frac{u_i-u_{i-1}}{b^2}\right)=0
	\end{array}
$$
Que deviennent $L$ et les équations d'Euler-Lagrange dans la limite où $b\longrightarrow 0$ ?
\begin{itemize}
	\item $\frac{m}{b}\longrightarrow \mu$ la masse linéique.
	\item La loi de Hooke stipule que dans la limite élastique, l'allongement par unité de longueur est proportionnel à la force appliquée $F=Ya$, où $Y$ est le module d'Young et $a$ l'allongement par unité de longueur.
\end{itemize}
Ici, on a :
$$
	F=k(u_{i+1}-u_i)=kb\frac{u_{i+1}-u_i}{b}
$$
L'allongement linéique est $\frac{u_{i+1}-u_i}{b}$, donc $kb\longrightarrow Y$. Le Lagrangien devient donc :
$$
	L=\sum bL_i\longrightarrow\bigintsss \!\dif z\left(\frac{1}{2}\mu \dot{u}^2-\frac{1}{2}Y\left(\frac{\partial u}{\partial z}\right)^2\right) = \bigintsss\!\dif z\,\mathcal{L}\left(u,\dot{u},\frac{\partial u}{\partial z}\right)
$$

$\mathcal{L}$, la densité lagrangienne, dépend d'une fonction $u$ ainsi que de ses dérivées temporelle $\partial_t u$ et spatiale $\partial_z u$. On parle alors de fonctionnelle.

Les équations d'Euler-Lagrange se transforment en équation de propagation :
$$
	\mu \ddot{u}-Y\frac{\partial^2u}{\partial z^2}=0
$$

\subsubsection{Densité lagrangienne et équation de champ}
La généralisation en cas tridimensionnel est immédiate, \emph{i.e.} pour un champ $u(x,y,z,t)$, on a :
\begin{itemize}
	\item une densité lagrangienne $\mathcal{L}(u,\partial_x u,\partial_y u,\partial_z u,\partial_t u)$,
	\item un Lagrangien $L=\bigintssss \!\dif x\,\dif y\,\dif z\, \mathcal{L}$,
	\item et un principe de Hamilton $\delta\!\!\bigintssss\!\dif t\, L = 0$.
\end{itemize}
À partir de cela, il est possible de déterminer l'équation correspondant à la dynamique du champ.
$$
	\begin{array}{r@{\;}l}
		0&=\delta\!\!\bigintssss\!\dif t\,\dif x\,\dif y\,\dif z\,\mathcal{L}(u,\partial_x u,\partial_y u,\partial_z u,\partial_t u)\\
			&=\bigintsss\!\dif t\,\dif x\,\dif y\,\dif z \left(\frac{\partial \mathcal{L}}{\partial u}\delta u
				+ \frac{\partial \mathcal{L}}{\partial(\partial_x u)}\delta(\partial_x u)
				+ \frac{\partial \mathcal{L}}{\partial(\partial_y u)}\delta(\partial_y u)
				+ \frac{\partial \mathcal{L}}{\partial(\partial_z u)}\delta(\partial_z u)
				+ \frac{\partial \mathcal{L}}{\partial(\partial_t u)}\delta(\partial_t u)\right)
	\end{array}
$$
Or,
$$
	\bigintsss_{t_1}^{t_2}\!\dif t\frac{\partial \mathcal{L}}{\partial(\partial_t u)}\delta(\partial_t u)
		=\underbrace{\left[\frac{\partial \mathcal{L}}{\partial(\partial_t u)}\delta u\right]^{t_2}_{t_1}}_{\mathclap{=0\text{ car on fixe les extrémités}}}
			-\bigintsss^{t_2}_{t_1}\!\dif t\left(\frac{\partial}{\partial t}\left(\frac{\partial \mathcal{L}}{\partial(\partial_t u)}\right)\right)\delta u
$$
soit, en suivant le même raisonnement pour les termes en dérivées spatiales,
$$
	0=\bigintsss\!\dif t\,\dif x\,\dif y\,\dif z\left(\frac{\partial \mathcal{L}}{\partial u}
		-\frac{\partial}{\partial x}\left(\frac{\partial \mathcal{L}}{\partial(\partial_x u)}\right)
		-\frac{\partial}{\partial y}\left(\frac{\partial \mathcal{L}}{\partial(\partial_y u)}\right)
		-\frac{\partial}{\partial z}\left(\frac{\partial \mathcal{L}}{\partial(\partial_z u)}\right)
		-\frac{\partial}{\partial t}\left(\frac{\partial \mathcal{L}}{\partial(\partial_t u)}\right)\right)\delta u
$$
Comme le résultat est vrai pour tout $u$, on obtient finalement l'équation de la dynamique des champs :
$$
	\boxed{\frac{\partial \mathcal{L}}{\partial u}-\frac{\partial}{\partial x_i}\frac{\partial \mathcal{L}}{\partial(\partial_{x_i}u)}-\frac{\partial}{\partial t}\frac{\partial \mathcal{L}}{\partial(\partial_{t}u)}=0}
$$

Dans le cas du 2.1.1,
$$
	\begin{array}{c}
		\mathcal{L}(u,\partial_xu,\partial_tu)=\frac{1}{2}\mu(\partial_tu)^2-\frac{1}{2}Y(\partial_xu)^2\\
		\partial_t(\mu\partial_tu)-\partial_x(Y\partial_xu)=0=\mu\partial_t^2u-Y\partial_x^2u=0
	\end{array}
$$
On retrouve l'équation de propagation.

\begin{remarks}\hspace*{1em}
\begin{itemize}
	\item En notant que les variables temporelle et spatiales jouent un r\^ole équivalent, et en posant $x^0=t$, $x^1=x$, $x^2=y$ et $x^3=z$, on peut réécrire l'équation de la dynamique sous la forme suivante :
	$$
		\boxed{\frac{\partial}{\partial x^{\alpha}}\frac{\partial\mathcal{L}}{\partial\left(\frac{\partial u}{\partial x^{\alpha}}\right)}-\frac{\partial\mathcal{L}}{\partial u}=0}
	$$
	\item On peut bien entendu généraliser ce résultat au cas où le champ n'est pas un champ scalaire, mais par exemple un champ vectoriel. Le formalisme tensoriel est alors particulière bien adapté, avec le fait que l'intervalle entre deux évènements en mécanique classique s'écrit :
	$$
		\dif s^2=\dif x^2+\dif y^2+\dif z^2\text{, \emph{i.e.} }g_{0i}=g_{i0}=0\kern 1em \forall i
	$$
	\item On peut noter qu'un système à $N$ degrés de liberté donne $N$ équations du mouvement, alors qu'un système avec une infinité non dénombrable de degrés de liberté (soit un champ) donne une unique équation. La différence réside dans le fait que les $N$ équations d'Euler-Lagrange sont des équations différentielles ordinaires, alors que l'équation de la dynamique des champs est une EDP.
\end{itemize}
\end{remarks}

\subsubsection{Densité hamiltonienne et densité de moment}
En faisant l'analogie du passage du Lagrangien au Hamiltonien, ou en remarquant qu'une équation du second ordre peut se réécrire sous la forme de deux équations du premiers ordre, on introduit :
$$
	\Pi_t=\frac{\partial\mathcal{L}}{\partial\left(\frac{\partial u}{\partial t}\right)}
$$
la densité de moment, et 
$$
	\mathcal{H}(u,\partial_xu,\partial_yu,\partial_zu,\Pi_t)=\Pi_t\frac{\partial u}{\partial t}-\mathcal{L}
$$
la densité hamiltonienne.
Le Hamiltonien correspondant s'écrit alors :
$$
	H=\bigintssss\!\dif x\,\dif y\,\dif z\,\mathcal{H}(u,\partial_xu,\partial_yu,\partial_zu,\Pi_t)
$$

On arrive, de la même façon, aux équations de Hamilton :
$$
	\left\{\begin{array}{r@{\;}l}
		\frac{\partial \mathcal{H}}{\partial\Pi_t}&=\frac{\partial u}{\partial t}\\
		\frac{\partial \mathcal{H}}{\partial u}&=-\frac{\partial \Pi_t}{\partial t}
	\end{array}\right.
$$

\begin{remark}
	Tout comme on a $\frac{\dif A}{\dif t}=[H,A]$ si $A$ ne dépend pas explicitement du temps, on a :
	$$
		\frac{\dif A}{\dif t}=\bigintssss\!\dif x\,\dif y\,\dif z\,[\mathcal{H},A]
	$$
	qui est le point de départ pour la quantification des champs.
\end{remark}

\subsection{Lagrangien du champ électromagnétique}
Le but est de trouver $\mathcal{L}$ pour que l'équation de la dynamique corresponde aux équations de Maxwell. Comme ces dernières sont invariantes sous une transformation de Lorentz, on cherche $\mathcal{L}$ vérifiant cette condition. Or, 
$$
	S=\bigintssss\!\dif t\,\dif x\,\dif y\,\dif z\,\mathcal{L}=\bigintssss\!\dif^4x\,\mathcal{L}
$$
est un invariant, et $\dif^4x$ aussi, donc $\mathcal{L}$ doit être un invariant de Lorentz. On va donc l'écrire sous forme tensorielle.

Les équations de Maxwell sont en outre linéaires en $\vec{E}$ et en $\vec{B}$, donc $\mathcal{L}$ doit être quadratique en $\vec{E}$ et $\vec{B}$. On est amené à considérer $c^2\vec{B}^2-\vec{E}^2$, qu'on réécrit sous forme tensorielle avec
$$
	F_{\mu\nu}F^{\mu\nu}=\frac{2}{c^2}(c^2\vec{B}^2-\vec{E}^2)
$$

\begin{remark}
	$\vec{E}\cdot\vec{B}$ est aussi quadratique et invariant, mais on serait coincé par exemple dans les situations purement électrostatiques.
\end{remark}

$\mathcal{L}$ est une densité volumique d'énergie. On considère donc :
$$
	\mathcal{L}=-\frac{1}{4\mu_0}F_{\mu\nu}F^{\mu\nu}
$$

Vérifions-que cette forme de $\mathcal{L}$ convient :
$$
		\mathcal{L}=-\frac{1}{4\mu_0}g_{\mu\lambda}g_{\nu\sigma}
			(\partial^{\lambda}\phi^{\sigma}-\partial^{\sigma}\phi^{\lambda})
			(\partial^\mu \phi^\nu-\partial^\nu \phi^\mu)
			=\mathcal{L}(\partial\phi)
$$
$$
	\begin{array}{r@{\;}l}
		\frac{\partial\mathcal{L}}{\partial\left(\frac{\partial\phi^{\alpha}}{\partial x_{\beta}}\right)}
			&=-\frac{g_{\mu\lambda}g_{\nu\sigma}}{4\mu_0}\left(
			(\delta_{\beta\delta}\delta_{\alpha\sigma}-\delta_{\beta\sigma}\delta_{\alpha\lambda})F^{\mu\nu}+
			F^{\lambda\sigma}(\delta_{\beta\mu}\delta_{\alpha\nu}-\delta_{\beta\nu}\delta_{\alpha\mu})\right)\\
			&=-\frac{1}{4\mu_0}\left(
			(g_{\mu\beta}g_{\nu\alpha}-g_{\mu\alpha}g_{\nu\beta})F^{\mu\nu}+
			F_{\mu\nu}(\delta_{\beta\mu}\delta_{\alpha\nu}-\delta_{\beta\nu}\delta_{\alpha\mu})\right)\\
			&=-\frac{1}{4\mu_0}(F_{\beta\alpha}-F_{\alpha\beta}+F_{\beta\alpha}-F{\alpha\beta})\\
			&=\frac{1}{\mu_0}F_{\alpha\beta}\\
		\multicolumn{2}{l}{\frac{\partial\mathcal{L}}{\partial \phi^{\alpha}}=0}
	\end{array}
$$

L'équation de la dynamique est donc :
$$
	\frac{\partial}{\partial x_{\beta}}\frac{\partial\mathcal{L}}{\partial\left(\frac{\partial\phi^{\alpha}}{\partial x_{\beta}}\right)}-\underbrace{\frac{\partial\mathcal{L}}{\partial \phi^{\alpha}}}_{=0}=0
$$
soit
$$
	\partial^{\beta}\left(\frac{1}{\mu_0}F_{\alpha\beta}\right)=0
$$
\emph{i.e.} on retrouve $\partial_{\alpha}F^{\alpha\beta}=0$, les équations de Maxwell avec sources (mais ici, sans sources).

Les équations homogènes sont automatiquement satisfaites en introduisant le quadripotentiel : $F^{\mu\nu}=\partial^{\mu}\phi^{\nu}-\partial^{\nu}\phi^\mu$.

$$
	\begin{array}{r@{\;}l}
		\partial_\alpha F_{\beta\gamma} + \partial_\beta F_{\gamma\alpha} + \partial_\gamma F_{\alpha\beta}
		&= \partial_{\alpha}(g_{\beta\mu}g_{\gamma\nu}F^{\mu\nu})
			+ \partial_{\beta}(g_{\gamma\mu}g_{\alpha\nu}F^{\mu\nu})
			+ \partial_{\gamma}(g_{\alpha\mu}g_{\beta\nu}F^{\mu\nu})\\
		&= g_{\beta\mu}g_{\gamma\nu}\partial_\alpha (\partial^{\mu}\phi^{\nu}-\partial^{\nu}\phi^\mu)
			+ g_{\gamma\mu}g_{\alpha\nu}\partial_{\beta} (\partial^{\mu}\phi^{\nu}-\partial^{\nu}\phi^\mu)
			+ g_{\alpha\mu}g_{\beta\nu}\partial_{\gamma}(\partial^{\mu}\phi^{\nu}-\partial^{\nu}\phi^\mu)\\
		&= \partial_{\alpha}\partial_{\beta}\phi_{\gamma}
			- \partial_{\alpha}\partial_{\gamma}\phi_{\beta}
			+ \partial_{\beta}\partial_{\gamma}\phi_{\alpha}
			- \partial_{\beta}\partial_{\alpha}\phi_{\gamma}
			+ \partial_{\alpha}\partial_{\gamma}\phi_{\beta}
			- \partial_{\gamma}\partial_{\beta}\phi_{\alpha}\\
		 &= 0
	\end{array}
$$

\subsection{Lagrangien du système \{champs + particules\}}
On a déjà écrit le Lagrangien pour une particule (ou un système de particules), et on passe à la description volumique en remplaçant 
$$
	\begin{array}{r@{\;}ccl}
		&q&\text{ par }&\rho\defeq q_i\delta(\vec{r}-\vec{r}_i)\\
		\text{et }&q\frac{\dif x^\alpha}{\dif \tau}&\text{ par }&\rho U^\alpha=J^\alpha
	\end{array}
$$

L'action s'écrit alors :
$$
	S=S_{\text{particule libre}}+\bigintsss\!\dif t\,\dif x\,\dif y\,\dif z\left(-\frac{1}{4\mu_0}F_{\mu\nu}F^{\mu\nu}-A^\mu J_\mu\right)
$$
avec 
$$
	S_{\text{particule libre}}=\sum_{\text{particules}}\bigintsss\!\dif\tau\left(-mc\sqrt{g_{\alpha\beta}\frac{\dif x^\alpha}{\dif\tau}\frac{\dif x^\beta}{\dif\tau}}\right)
$$

Retrouve-t-on les équations de Maxwell avec sources et les équations de Lorentz ?
$$
	\begin{array}{c}
		\frac{\partial\mathcal{L}}{\partial\left(\frac{\partial A^{\alpha}}{\partial x_{\beta}}\right)}=\frac{1}{\mu_0}F_{\alpha\beta}\\
		\frac{\partial \mathcal{L}}{\partial A^\alpha}=-J_\alpha
	\end{array}
$$

L'équation de la dynamique prend la forme suivante :
$$
	\begin{array}{r@{\;}l}
		0&=\frac{\partial}{\partial x_{\beta}}\frac{\partial\mathcal{L}}{\partial\left(\frac{\partial A^\alpha}{\partial x_{\beta}}\right)}-\frac{\partial\mathcal{L}}{\partial A^\alpha}\\
		&=\partial^\beta\frac{1}{\mu_0}F_{\alpha\beta}+J_\alpha\\
		&=-\partial^\beta\frac{1}{\mu_0}F_{\beta\alpha}+J_\alpha
	\end{array}
$$
$$
	\boxed{\partial^\beta F_{\beta\alpha}=\mu_0 J_\alpha}
$$
On a donc retrouvé les équations de Maxwell avec sources. Les équations de Maxwell homogènes restent vérifiées. L'équation de Lorentz est également satisfaite car on a montré que 
$$
aze	
$$
{\txt et le terme supplémentaire $-\frac{1}{4\mu_0}F_{\mu\nu}F^{\mu\nu}$ ne fait pas intervenir les $(x^\alpha,U^\alpha)$, donc la démonstration précédente est toujours valable.}

\begin{remarks}\hspace*{1em}
	\begin{itemize}
		\item La densité lagrangienne n'est pas unique, tout comme en mécanique analytique classique, et on peut ajouter n'importe quel terme de la forme $\partial_\mu F^\mu(x^\alpha,A^\beta)$.
		\item On peut poursuivre le travail pour obtenir la forme hamiltonienne.
	\end{itemize}
\end{remarks}

\section{Lois de conservation}
\subsection{Théorème de Noether}
Dans le cas discret, on considère une transformation infinitésimale $q_i\longrightarrow q_i'$ et on sait que si $\delta L=\left.\frac{\dif L}{\dif \epsilon}\right|_{\epsilon=0}=0$, on est capable d'exhiber une quantité conservée $K(q,\dot{q})$, \emph{i.e.}
$$
	\frac{\dif K(q,\dot{q})}{\dif t}=0
$$
Le pendant en théorie des champs est que l'on effectue des transformations infinitésimales sur les champs $\psi$, que le rôle de $\dif t$ est joué par $\dif^4x$, et que si l'on est capable de trouver
$$
	\phi^\alpha(\psi,x) \text{ tel que } \left.\frac{\dif \mathcal{L}}{\dif\epsilon}\right|_{\epsilon=0}=\partial_\alpha \phi^\alpha
$$
alors
$$
	\partial_\alpha G^\alpha=0 \text{ avec } G^\alpha=\frac{\partial \mathcal{L}}{\partial\left(\frac{\partial\psi_\rho}{\partial x^\alpha}\right)}\delta\psi_\rho-\phi^\alpha
$$
Le pendant de la charge de Noether est donc un courant dont la quadri-divergence est nulle.

\subsection{Applications}
\subsubsection{Invariance de jauge et conservation de la charge}
L'action s'écrit :
$$
	S=\underbrace{S_{\text{particule libre}}}_{\text{indépendant de la jauge}}
	+\bigintsss\!\dif t\,\dif x\,\dif y\,\dif z\left(\vphantom{\frac{1}{4}}\right.
	-\underbrace{\frac{1}{4\mu_0}F_{\mu\nu}F^{\mu\nu}}_{\text{indépendant de la jauge}}
	-\underbrace{A^\mu J_\mu}_{\text{dépend de la jauge}}\left.\vphantom{\frac{1}{4}}\right)
$$

Considérons la transformation infinitésimale suivante :
$$
	\left\{\begin{array}{r@{\longrightarrow}l}
		\frac{\phi}{c}&\frac{\phi}{c}-\partial_t\Lambda\\
		\vec{A}&\vec{A}+\nab\Lambda
	\end{array}\right.
$$

On s'intéresse uniquement à la partie dépendant de la jauge :
$$
	\begin{array}{r@{\;}l}
		-A^\mu J_\mu \longrightarrow & -A^\mu J_\mu+(\partial_t\Lambda)J_0-\nab\Lambda\cdot(-\vec{j})\\
			=&-A^\mu J_\mu+\underbrace{\partial_\mu(\Lambda J^\mu)}_{\mathclap{\text{de la forme }\partial_\mu\Gamma^\mu\text{, ne contribue pas}}}-\Lambda\partial_\mu J^\mu\\
	\end{array}
$$
donc, d'après le théorème de Noether,
$$
	\phi^\alpha\defeq-J^\alpha
$$
et la quadri-divergence de $G$ est nulle, avec
$$
	G^\alpha=\underbrace{\frac{\partial \mathcal{L}}{\partial\left(\frac{\partial A_\rho}{\partial x^\alpha}\right)}}_{=0}\delta A_\rho-\phi^\alpha
$$
\emph{i.e.} $G^\alpha=J^\alpha$ et le théorème assure que $\partial_\alpha J^\alpha=0$, on retrouve l'équation de conservation de la charge.

\subsubsection{Tenseur énergie-impulsion du champ électromagnétique}

{\small \it cf. TD}
\end{document}