\documentclass[11pt,a4paper]{report}
\usepackage[utf8]{inputenc}
\usepackage[frenchb]{babel}
\usepackage[T1]{fontenc}
\usepackage[intlimits]{amsmath}
\usepackage{amsthm}
\usepackage{amsfonts}
\usepackage{amssymb}
\usepackage[left=2cm,right=2cm,top=2cm,bottom=2cm]{geometry}
\usepackage{tabularx}
\usepackage[pdftex,colorlinks=true,pdfstartview=FitB,hyperindex=true]{hyperref}
\usepackage{bigints}

\usepackage{graphicx}
\usepackage[hang,small,bf]{caption}
\usepackage{float}

%\usepackage{calc}
%\newlength\dlf
%\newcommand\alignedbox[2]{
%  % #1 = before alignment
%  % #2 = after alignment
%  &
%  \begingroup
%  \settowidth\dlf{$\displaystyle #1$}
%  \addtolength\dlf{\fboxsep+\fboxrule}
%  \hspace{-\dlf}
%  \boxed{#1 #2}
%  \endgroup
%}


\newenvironment{mat}[1][1]
{\renewcommand*{\arraystretch}{#1} \begin{pmatrix}}
{\end{pmatrix}}


\theoremstyle{plain}
\newtheorem*{theorem}{Théorème}
\newtheorem*{postulat}{Postulat}

\theoremstyle{definition}
\newtheorem*{ex}{Exemple}
\newtheorem*{cons}{Conséquence}
\newtheorem*{corol}{Corollaire}
\newtheorem*{conc}{Conclusion}
\newtheorem*{app}{Application}

\theoremstyle{remark}
\newtheorem*{remark}{Remarque}
\newtheorem*{remarks}{Remarques}
\newtheorem*{rappel}{Rappel}
\newtheorem*{exo}{Exercice}


\newcommand{\txt}{
\everymath{\textstyle}
}

\newcommand{\para}{
\ensuremath
\mbox{\textbf{\tiny /\!/}}
}

\newcommand{\nab}{
\ensuremath
\vec{\nabla}
}

\newcommand{\bigiiintsss}{
\bigintsss\kern -6pt \bigintsss_V\kern -6pt \bigintsss
}

\newcommand{\dif}{
\mathrm{d}
}

\author{Christophe \textsc{Winisdoerffer}}
\title{Électrodynamique}
\date{}

\renewcommand{\thesection}{\arabic{section}}
\setcounter{secnumdepth}{3}

\everymath{\displaystyle}
\renewcommand*{\arraystretch}{1.7}

\begin{document}
 	



\thispagestyle{empty}
\newgeometry{left=3cm,right=3cm,top=8cm,bottom=2cm}
\makeatletter

\begin{flushleft}
	\Huge \@title
\end{flushleft}
\hrule height 6pt
\begin{flushright}
	Master sciences de la matière\\
	Semestre 1, ENS Lyon
\end{flushright}

\vfill
\noindent
\begin{tabularx}{\textwidth}{l X r}
	Cours de :  & & Retranscrit par : \\
	\@author & & Simon \textsc{Zugmeyer} \\
	\href{mailto:cwinisdo@ens-lyon.fr}{\tt cwinisdo@ens-lyon.fr} & &
\end{tabularx}

\makeatother
\restoregeometry

\tableofcontents
\pagebreak

\chapter{Les équations de Maxwell}


{\small \it Note : les quatre premières sections sont des polycopiés distribués par le professeur.}
\section{Introduction}
\section{Équations de Maxwell}
\section{Les invariants de Maxwell-Lorenz}
\section{Équations de Maxwell et changement de référentiel}
\subsection{Transformation de Galilée}

On considère deux référentiels $\mathcal{R}$ et $\mathcal{R}'$ en translation rectiligne uniforme l'un par rapport à l'autre, et pour lesquels on a choisi les origines spatiales et temporelles de telle façon que le système de coordonnées s'écrive :
\begin{align*}
	t'&=t\\
	\vec{r'}&=\vec{r}-\vec{v}t, \left\{ \begin{array}{r@{\;}l}
			x'&=x-v_{x}t \\
			y'&=y-v_{y}t \\
			z'&=z-v_{z}t \\
		\end{array} \right.
\end{align*}
On a alors :
\begin{align*}
	\frac{\partial}{\partial t}&=\frac{\partial t'}{\partial t} \frac{\partial}{\partial t'}
			+\frac{\partial x'}{\partial t} \frac{\partial}{\partial x'}
			+\frac{\partial y'}{\partial t} \frac{\partial}{\partial y'}
			+\frac{\partial z'}{\partial t} \frac{\partial}{\partial z'}\\
	&=\frac{\partial}{\partial t'} - v_{x} \frac{\partial}{\partial x'}
			- v_{y} \frac{\partial}{\partial y'}
			- v_{z} \frac{\partial}{\partial z'}\\
	&=\frac{\partial}{\partial t'} - \vec{v}\cdot \vec{\nabla '}\\[10pt]
	\frac{\partial}{\partial x}&=\frac{\partial t'}{\partial x} \frac{\partial}{\partial t'}
			+\frac{\partial x'}{\partial x} \frac{\partial}{\partial x'}
			+\frac{\partial y'}{\partial x} \frac{\partial}{\partial y'}
			+\frac{\partial z'}{\partial x} \frac{\partial}{\partial z'}\\
	&=\frac{\partial}{\partial x'}\\
	\frac{\partial}{\partial y}&=\frac{\partial}{\partial y'}\\
	\frac{\partial}{\partial z}&=\frac{\partial}{\partial z'}
\end{align*}
et donc 
\begin{align*}
	\nab\cdot\vec{E}(x,y,z,t)&=\vec{\nabla '}\cdot\vec{E}(x(x',y',z',t'),y(x',y',z',t'),z(x',y',z',t'),t(x',y',z',t'))\\ 
	\nab\cdot\vec{E}&=\vec{\nabla '}\cdot\vec{E}
\end{align*}

\subsubsection*{Maxwell-Faraday}
\begin{align*}
	\nab\times\vec{E}(\vec{r},t)&=-\frac{\partial \vec{B}}{t}\\
		&=\vec{\nabla '}\times\vec{E}\\
		&=-\frac{\partial \vec{B}}{t'} + (\vec{v}\cdot\vec{\nabla '})\vec{B}\\[15pt]
	\nab\times(\vec{a}\times\vec{b})&=\vec{a}(\nab\cdot\vec{b})
			-\vec{b}(\nab\cdot\vec{a})
			+(\vec{b}\cdot\nab)\vec{a}
			-(\vec{a}\cdot\nab)\vec{b}\\
	(\vec{v}\cdot\vec{\nabla '})\vec{B}&=\underbrace{\vec{v}(\vec{\nabla '}\cdot\vec{B})}_{=0}
			-\underbrace{\vec{v}(\vec{\nabla '}\cdot\vec{B})}_{=0}
			+\underbrace{(\vec{B}\cdot\vec{\nabla '})\vec{v}}_{=0}
			-\vec{\nabla '}\times(\vec{v}\times\vec{B})\\
	\vec{\nabla '}\times\vec{E}&=-\frac{\partial \vec{B}}{t'}-\vec{\nabla '}\times(\vec{v}\times\vec{B})
\end{align*}
\boxed{\vec{\nabla '}\times(\vec{E}+\vec{v}\times\vec{B})=-\frac{\partial \vec{B}}{t'}}		

\subsubsection*{Maxwell-Gauss}
\begin{align*}
{\nabla}\cdot\vec{E}=\boxed{\frac{\rho}{\epsilon_0}=\vec{\nabla '}\cdot\vec{E}}
\end{align*}

\subsubsection*{Maxwell-Ampère}
\begin{align*}
	\nab\times\vec{B}&=\mu_0\vec{j}+\mu_0\epsilon_0\frac{\partial \vec{E}}{\partial t}\\
		&=\vec{\nabla '}\times\vec{B}\\
		&=\mu_0\vec{j}+\mu_0\epsilon_0\frac{\partial \vec{E}}{\partial t'}-\mu_0\epsilon_0(\vec{v}\cdot\vec{\nabla '})\vec{E}\\[15pt]
	(\vec{v}\cdot\vec{\nabla '})\vec{E}&=\underbrace{\vec{v}(\vec{\nabla '}\cdot\vec{E})}_{=\frac{\rho}{\epsilon_0}}
			-\underbrace{\vec{v}(\vec{\nabla '}\cdot\vec{E})}_{=0}
			+\underbrace{(\vec{E}\cdot\vec{\nabla '})\vec{v}}_{=0}
			-\vec{\nabla '}\times(\vec{v}\times\vec{E})
\end{align*}
\boxed{\vec{\nabla '}\times(\vec{B}+\frac{\vec{v}\times\vec{E}}{c^2})=\mu_0(\vec{j}-\rho\vec{v})+\epsilon_0\mu_0\frac{\partial \vec{E}}{t'}}	


\subsubsection*{Conservation de la charge}
\begin{align*}
	\frac{\partial\rho}{\partial t}+\nab\cdot\vec{j}=\boxed{0=\frac{\partial\rho}{\partial t'}+\vec{\nabla '}\cdot(\vec{j}-\rho\vec{v})}
\end{align*}

\subsubsection*{Conclusion}
\begin{itemize}
	\item conservation charge $\left\{ \begin{array}{r@{\;}l}
					\rho '&=\rho\\
					\vec{j'}&=\vec{j}-\rho\vec{v}
			\end{array} \right.$
	\item Maxwell-Gauss $\vec{E '}=\vec{E}$
	\item Maxwell-flux $\vec{B '}=\vec{B}$
	\item Maxwell-Faraday $\vec{E '}=\vec{E}+\vec{v}\times\vec{B} si \vec{B'}=\vec{B}$
	\item Maxwell-Ampère $\vec{B '}=\vec{B}+\frac{\vec{v}\times\vec{E}}{c^2} si \vec{E'}=\vec{E}$
\end{itemize}

Les équations de Maxwell ne sont donc pas invariantes sous une transformation de Galilée.

\subsection{Transformation de Lorenz}
On considère deux repères $\mathcal{R}$ et $\mathcal{R}'$ en translation rectiligne uniforme l'un par rapport à l'autre, avec un choix d'origines tel que :
$\left\{ \begin{array}{r@{\;}l}
		ct'&=\gamma ct-\beta\gamma x\\
		x'&=-\beta\gamma ct+\gamma x\\
		y'&=y\\
		z'&=z
\end{array} \right.$
Dans ce cas,
\begin{align*}
	\frac{\partial}{\partial t}&=\frac{\partial t'}{\partial t} \frac{\partial}{\partial t'}
			+\frac{\partial x'}{\partial t} \frac{\partial}{\partial x'}
			+\frac{\partial y'}{\partial t} \frac{\partial}{\partial y'}
			+\frac{\partial z'}{\partial t} \frac{\partial}{\partial z'}\\
		&=\gamma\frac{\partial}{\partial t'}-\beta\gamma c\frac{\partial}{\partial x'}\\[10pt]
	\frac{\partial}{\partial x}&=-\frac{\beta\gamma}{c}\frac{\partial}{\partial t'}+\gamma\frac{\partial}{\partial x'}\\
	\frac{\partial}{\partial y}&=\frac{\partial}{\partial y'}\\
	\frac{\partial}{\partial z}&=\frac{\partial}{\partial z'}
\end{align*}

\subsubsection*{Maxwell-Faraday}
\begin{align*}
	\nab\times\vec{E}&=-\frac{\partial \vec{B}}{t}
\end{align*}
\indent$
	\left\{ \begin{array}{l}
		\partial_yE_z-\partial_zE_y=-\partial_tB_x\\
		\partial_zE_x-\partial_xE_z=-\partial_tB_y\\
		\partial_xE_y-\partial_yE_x=-\partial_tB_z\\
	\end{array} \right.\\[5pt]\indent
	\left\{
	\begin{array}{r@{\;}l}
		\partial_{y'}E_z-\partial_{z'}E_y&=-\gamma\partial_{t'}B_x+\beta\gamma c\partial_{x'}B_x\\
		\partial_{z'}E_x-\partial_{x'}(\gamma E_z)+\frac{\beta\gamma}{c}\partial_{t'}E_z&=-\gamma\partial_{t'}B_y+\beta\gamma c\partial_{x'}B_y\\
		\partial_{x'}E_y-\frac{\beta\gamma}{c}\partial_{t'}E_y-\partial_{y'}E_x&=-\gamma\partial_{t'}B_z+\beta\gamma c\partial{x'}B_z\\
	\end{array} \right.\\[10pt]\indent
	\begin{array}{r@{\;}l}
		\text{Maxwell-Flux donne : }{\nabla}\cdot\vec{B}=0&=\partial_xBx+\partial_yB_y+\partial_zB_z\\
		&=-\frac{\beta\gamma}{c}\partial_{t'}B_x+\gamma\partial_{x'}B_x+\partial_{y'}B_y+\partial_zB_z\\
		\beta\gamma c\partial_{x'}B_x&=\gamma\beta^2\partial_{t'}B_x-\beta c\partial_{y'}B_y-\beta c\partial_{z'}B_z\\[10pt]\indent
		\gamma&=\frac{1}{\sqrt{1-\beta^2}}\\
		\gamma(1-\beta^2)&=\frac{1}{\gamma}
	\end{array}\\[10pt]\indent
	\left\{
	\begin{array}{c@{-}c@{\;}l}
		\partial_{y'}(\gamma(E_z+\beta cB_y))&\partial_{z'}(\gamma(E_y-\beta cB_z))&=-\partial_{t'}B_x\\
		\partial_{z'}E_x&\partial_{x'}(\gamma(E_z+\beta cB_y))&=-\partial_{t'}(\gamma(B_y+\frac{\beta}{c}E_z))\\
		\partial_{x'}(\gamma(E_y-\beta cB_z))&\partial_{y'}E_x&=-\partial_{t'}(\gamma(B_z-\frac{\beta}{c}E_y))\\
	\end{array} \right.\\
$

En faisant le même travail pour les autres équations de Maxwell, on s'aperçoit qu'elles sont invariantes pour :\\
\indent$
\left\{ \begin{array}{r@{\;}l}
	c\rho '&=\gamma c\rho-\gamma\vec{\beta}\cdot\vec{j}\\
	\vec{j'_{\para}}&=-\vec{\beta}\gamma c\rho+\gamma\vec{j_{\para}}\\
	\vec{j'_{\perp}}&=\vec{j_{\perp}}
\end{array} \right.\\[10pt]\indent
\left\{ \begin{array}{r@{\;}l}
	\vec{E'_{\para}}&=\vec{E_{\para}}\\
	c\vec{B'_{\para}}&=c\vec{B_{\para}}\\
	\vec{E'_{\perp}}&=\gamma(\vec{E_{\perp}}+\vec{v}\times\vec{B_{\perp}})\\
	c\vec{B'_{\perp}}&=\gamma(c\vec{B_{\perp}}-\frac{\vec{v}\times\vec{E_{\perp}}}{c})
\end{array} \right.\\
$

%\[
%  \begin{array}{r@{\;}l}
%    f(x) & = a \\
%    g(x) & = ax + b \\
%    h(x) & = ax^2 + bx + c \hspace*{3em}
%      \smash{\left.\begin{array}{@{}c@{}}\\ \\ \\ \end{array}\right\}} \\
%    i(x) & = ax^3 + bx^2 + cx + d \\
%    j(x) & = ax^4 + bx^3 + cx^2 + dx + e
%  \end{array}
%\]

Remarques : 
\begin{itemize}
	\item On reconna\^it dans la transformation de $(c\rho,\vec{j})$ celle correspondant à un quadrivecteur
	\item Transformation inverse : permuter les variables primées et non primées, et changer le signe de $\beta$
	\item C'est Lorenz qui choisit l'invariance des équations de Maxwell sans les sources en 1904, et Poincaré avec les sources en 1905 en exploitant la transformation de $(c\rho,\vec{j})$
\end{itemize}

\section{Potentiels et jauges}
\subsection{Potentiels}
	Les équations de Maxwell fournissent quatre équation aux dérivées partielles couplées. Les deux équations homogènes ({\it i.e.} sans source, Maxwell-flux et Maxwell-Faraday) peuvent être identiquement résolues en introduisant le potentiel vecteur $\vec{A}(\vec{r},t)$ et le potentiel scalaire $\Phi(\vec{r},t)$ tels que :
	\begin{align*}
		\vec{B}&=\nab\times\vec{A}\\
		\vec{E}&=-\nab\Phi-\partial_t\vec{A}\\
		\text{car } \nab\cdot(\nab\times\vec{Z})&=0\\
		\text{et }\nab\times(\nab Z)&=\vec{0}
	\end{align*}
	
	Les deux autres équations deviennent alors :
	
	\begin{align*}
		\nab\vec{E}=\frac{\rho}{\epsilon_0}&=\nab\cdot(-\nab\Phi-\partial_t\vec{A})\\
		&=-\nab^2\Phi-\partial_t(\nab\cdot\vec{A})\\[10pt]
		\nab\times\vec{B}&=\mu_0\vec{j}+\mu_0\epsilon_0\partial_t\vec{E}\\
		&=\nab\times(\nab\times\vec{A})\\
		&=-\nab^2+\nab(\nab\cdot\vec{A})\\
		&=\mu_0\vec{j}+\mu_0\epsilon_0\partial_t(-\nab\phi-\partial_t\vec{A})\\
		&=\mu_0\vec{j}+\frac{1}{c^2}\nab(-\partial_t\Phi)-\frac{1}{c^2}\partial_t^2\vec{A}
	\end{align*}
	
	soit 
	\boxed{\left\{ \begin{array}{{r@{\;}l}}
		\nab^2\vec{A}-\frac{1}{c^2}\partial_t^2\vec{A}&=-\mu_0\vec{j}+\nab(\nab\cdot\vec{A}+\frac{1}{c^2}\partial_t\Phi)\\
		\nab^2\Phi-\frac{1}{c^2}\partial_t^2\Phi&=-\frac{\rho}{\epsilon_0}-\partial_t(\nab\cdot\vec{A}+\frac{1}{c^2}\partial_t\Phi) 
	\end{array}\right.}\\
	
	On est passé de quatre EDP couplées du premier ordre à deux EDP couplées du second ordre, ce qui, en soit, n'est pas utile. Cependant, on a une liberté dans le choix de $\Phi$ et $\vec{A}$, seuls $\vec{E}$ et $\vec{B}$ sont mesurables. En effet, avec la transformation :\\
\indent
	$ 
		\begin{array}{r@{\;}l}
				\vec{A'} &=\vec{A}+\nab\lambda(\vec{r},t)\\
			\Phi'&=\Phi-\partial_t\lambda(\vec{r},t)
		\end{array}\\
	$	
	Si $\lambda$ est suffisament régulier,\\
\indent
	$
		\begin{array}{r@{\;}l}
			\nab\times(\vec{A')}&=(\vec{A}+\nab\times(\nab\lambda)\\ 
			&=\vec{B}
		\end{array}\\[5pt]\indent
		\begin{array}{r@{\;}l}
			-\nab\Phi-\partial_{t'}\vec{A'}&=-\nab\Phi-\partial_t\vec{A}+\nab\partial_t\lambda-\partial_t\nab\lambda\\
			&=\vec{E}
		\end{array}
	$
	
	$(\vec{A'},\Phi')$ correspondent aux m\^emes champs $(\vec{E},\vec{B})$ que $(\vec{A},\Phi)$. Cette liberté s'appelle \emph{invariance de jauge}, et la transformation précédente \emph{transformation de jauge}. Seuls $\vec{E}$ et $\vec{B}$ sont observables ({\it i.e.} mesurables avec un détecteur local, en regardant le mouvement d'une charge), alors que $\Phi$ et $\vec{A}$ sont des intermédiaires de calcul. On se sert donc de cette liberté pour simplifier les équations.
\begin{remark}
	On ne perd rien en écrivant $\vec{E}$ et $\vec{B}$ avec $\vec{A}$ et $\Phi$. Si $\vec{E}$ et $\vec{B}$ sont solution des équations de Maxwell avec les conditions aux limites, alors on peut toujours trouver $\vec{A}$ et $\Phi$ vérifiant :
	\indent
	$
		\left\{ \begin{array}{r@{\;}l}
			\vec{B}&=\nab\times\vec{A}\\
			\vec{E}&=-\nab\Phi-\partial_t\vec{A}\\
		\end{array} \right.
	$
\end{remark}

\subsection{Jauges}
\subsubsection{Jauge de Lorenz}
	La jauge de Lorenz consiste à imposer \boxed{\nab\cdot\vec{A}+\frac{1}{c^2}\partial_t\Phi=0}
	Dans ce cas, les équations deviennent :
	\indent
	$
		\left\{ \begin{array}{c@{-}c@{\;}l}
			\nab^2\Phi&\frac{1}{c^2}\partial_t^2\Phi&=-\frac{\rho}{\epsilon_0}\\
			\nab^2\vec{A}&\frac{1}{c^2}\partial_t^2\vec{A}&=-\mu_0\vec{j}
		\end{array} \right.
	$
	\emph{i.e.} les équations sont découplées.

	\begin{theorem}
		On peut toujours trouver un couple $(\vec{A},\Phi)$ qui satisfait la jauge de Lorenz.
	\end{theorem}
	
	\begin{proof}
		Soit $(\vec{A},\Phi)$ qui ne satisfait pas la jauge de Lorenz. On cherche $(\vec{A'},\Phi')$ relié à $(\vec{A},\Phi)$ par une transformation de jauge tel qu'il satisfasse la jauge de Lorenz.\\
\indent
	$ 
		\begin{array}{r@{\;}l}
				\vec{A'} &=\vec{A}+\nab\lambda(\vec{r},t)\\
			\Phi'&=\Phi-\partial_t\lambda(\vec{r},t)
		\end{array}\\[10pt]\indent
		\begin{array}{r@{\;}l}
			\nab\cdot\vec{A'}+\frac{1}{c^2}\partial_t\Phi'&=0\\
			&=\nab\cdot(\vec{A}+\nab\lambda)+\frac{1}{c^2}\partial_t(\Phi-\partial_t\lambda)\\
			&=\nab\cdot\vec{A}+\nab^2\lambda+\frac{1}{c^2}\partial_t\Phi-\frac{\partial_t}{c^2}\lambda
		\end{array}
	$	
	\medskip
	
	Il faut résoudre $\nab^2\lambda-\frac{1}{c^2}\partial_t\lambda=-(\nab\cdot\vec{A}+\frac{1}{c^2}\partial_t\Phi)$ qui admet toujours une solution. \qedhere
	\end{proof}
	
	
	\begin{remarks} \hspace{1cm}
		\begin{itemize}
			\item La simplicité des équations sur les potentiels est telle qu'on peut aisément les résoudre à l'aide de la technique des fonctions de Green (\emph{cf.} potentiels de Liénard-Wiechert).
			\item Cette jauge est dite covariante, car on peut montrer (\emph{cf.} plus loin que si elle est satisfaite dans $\mathcal{R}$, alors elle est satisfaite dans $\mathcal{R}'$ en translation rectiligne uniforme par rapport à $\mathcal{R}$. Elle est donc particulièrement adaptée pour une description relativiste de l'électromagnétique.
			\item On remarque la relation suivante :\\
			 $-\square(\underbrace{\nab\cdot\vec{A}+\frac{1}{c^2}\partial_t\Phi}_{\text{jauge}})=\mu_0(\underbrace{\partial_t\rho+\nab\cdot\vec{j}}_{\text{conservation  charge}})$
			\hspace{1cm} où
			 $\square=\nab^2-\frac{1}{c^2}\partial_t^2$
		\end{itemize}
	\end{remarks}
	
	\subsubsection{Jauge de Coulomb}
	
	La jauge de Coulomb s'écrit \boxed{\nab\cdot\vec{A}=0}
	Les équations deviennent :\\
	\indent
	$\left\{ \begin{array}{l}
		\nab^2\Phi=-\frac{\rho}{\epsilon_0} \text{ \hspace{1cm} dont la solution est } \Phi(\vec{r},t)=\frac{1}{4\pi\epsilon_0}\int_Vd^3\vec{r'}\frac{\rho(\vec{r},t)}{\parallel\vec{r}-\vec{r'}\parallel}\\
		\nab^2-\frac{1}{c^2}\partial_t^2\vec{A}=-\mu_0\vec{j}+\nab(\frac{1}{c^2}\partial_t\Phi)	
	\end{array} \right.
	$
	
	\begin{remarks}
		\hspace{1pt}
		\begin{itemize}
			\item On sait que les signaux électromagnétiques se propagent dans le vide à la vitesse $c$, alors que $\Phi$ se "propage" instantanément. Il est à noter, à nouveau, que seuls $\vec{E}$ et $\vec{B}$ sont observables.
			\item Cette jauge n'est pas covariante, car la variable temporelle n'intervient pas.
		\end{itemize}
	\end{remarks}
	
	\begin{theorem}
		On peut toujours trouver un tel $\vec{A}$ dans un référentiel $\mathcal{R}$.
	\end{theorem}
	
	\begin{proof}
		Soit $(\vec{A},\Phi)$ qui ne satisfait pas la jauge de Coulomb. On cherche $(\vec{A'},\Phi')$ relié à $(\vec{A},\Phi)$ par une transformation de jauge tel que :\\
\indent
	$ 
		\begin{array}{r@{\;}l}
				\vec{A'} &=\vec{A}+\nab\lambda(\vec{r},t)\\
			\Phi'&=\Phi-\partial_t\lambda(\vec{r},t)
		\end{array}\\[10pt]\indent
		\nab\cdot\vec{A}=0=\nab\cdot\vec{A}+\nab^2\lambda
	$

	Il suffit de résoudre l'équation de Poisson $\nab^2\lambda=-\nab\cdot\vec{A}$ \qedhere
	\end{proof}
	
	
	\begin{remarks}\hspace{1pt}
		\begin{itemize}
			\item Cette jauge s'appelle aussi jauge électrostatique, car $\Phi$ vérifie le même type d'équation.
			\item Cette jauge s'appelle aussi jauge transverse :\\
			\indent $\begin{array}{r@{\;}l}
					\vec{j}=\vec{j_p}+\vec{j_t} \text{ avec }&\vec{j_p} \text{ longitudinal tel que } \nab\times\vec{j_p}=\vec{0}\\
					&\vec{j_t} \text{ transverse tel que } \nab\times\vec{j_t}=\vec{0}
			\end{array}\\[10pt]\indent
			\nab^2\vec{A}-\frac{1}{c^2}\partial_t^2\vec{A}=-\mu_0\vec{j_t}\\
			\text{\hspace{3cm} avec }\vec{j_t}=\frac{1}{4\pi}\nab\times\nab\times\int\frac{\vec{j}(\vec{r'},t}{\parallel\vec{r}-\vec{r'}\parallel}d^3\vec{r'}
			$
			\item Cette jauge s'appelle aussi jauge de radiation, car si on a un ensemble de charges dans un volume fini de l'espace, alors on peut montrer que le champ $\vec{E}$, $\vec{B}$ contient deux composantes, une en $\frac{1}{r}$ et l'autre en $\frac{1}{R^2}$ lorsque $R\rightarrow+\infty$. Si les charges ne sont pas accélérées, \emph{i.e.} c'est un problème de statique, alors seul le terme en $\frac{1}{r^2}$ subsiste. Le terme en $\frac{1}{R}$ est dit de radiation. Dans cette jauge, il provient exclusivement de l'équation sur $\vec{A}$ (car sur $\Phi$, c'est une équation de Poissson). C'est une jauge particulièrement bien adaptée pour la quantification du champ électromagnétique.
			\item Comme précédemment, on a :\\
					 $-\square(\underbrace{\nab\cdot\vec{A}}_{\text{jauge}})=\mu_0(\underbrace{\partial_t\rho+\nab\cdot\vec{j}}_{\text{conservation  charge}})$
			\hspace{1cm} où
			 $\square=\nab^2-\frac{1}{c^2}\partial_t^2$
		\end{itemize}
	\end{remarks}
\chapter{Formulation relativiste}


{\small \it Note : la première partie se trouve sur une feuille distribuée par le professeur.}
\section{Rappels d'analyse tensorielle}
\section{Formulation relativiste de l'électrodynamique}
\subsection{Postulats fondamentaux}

\begin{postulat}
	Les lois de la physique, lorsqu'elles sont formulées de manière adéquate, gardent la m\^eme forme dans un référentiel $\mathcal{R}$ et dans un référentiel $\mathcal{R}'$ en translation rectiligne uniforme par rapport à $\mathcal{R}$. Il y a invariance de la forme des équations.\\
	Contribution d'Einstein : la vitesse de propagation d'un signal électromagnétique dans le vide est universelle et indépendante du référentiel.
\end{postulat}

\begin{cons}
	Ce sont les équations de Maxwell qui sont invariantes, et cela a donc amené à abandonner la mécanique classique (Newtonienne) pour la reformuler afin qu'elle soit aussi invariante sous une transformation de Lorentz.
\end{cons}

\begin{postulat}
	Invariance de la charge : en accord avec les expériences, on postule que la charge q d'une particule ne dépend pas du référentiel galiléen considéré.
\end{postulat}

\begin{cons}
	L'invariance formelle des lois de la physique est automatiquement satisfaite si on les écrit sous forme temporelle (condition suffisante), sachant que les transformations à considérer sont les transformations du groupe de Lorentz, qui recouvre les translations spatiales et temporelles, les rotations, les renversements d'axe, et les boosts de Lorentz.
$$
	x'^{\mu}=\mathcal{L}^{\mu}_{\nu}x^{\mu} \text{ avec }\mathcal{L}^{\mu}_{\nu}=\begin{mat}
	\gamma & -\beta\gamma & \hspace*{0.2cm}0\hspace*{0.2cm} \\
	-\beta\gamma & \gamma & \hspace*{0.2cm}0\hspace*{0.2cm} \\
	0 & 0 & \hspace*{0.2cm}1\hspace*{0.2cm}
	\end{mat}
$$
\end{cons}


\subsection{Quadrivecteur courant}
{\txt
On cherche à construire un quadrivecteur à partir de $\rho$ et $\vec{j}$. Ainsi, on considère une charge $q$ dans un référentiel $\mathcal{R}$, que l'on modélise comme une distribution de charge $\rho$ dans un volume $dV=\frac{q}{\rho}$. Dans $\mathcal{R}'$ en translation rectiligne uniforme par rapport à $\mathcal{R}$, cette charge $q$ occupe $dV'$ et correspond à $\rho'$. D'après le postulat d'invariance de la charge, on a : $\rho dV=q=\rho' dV'$. Dans $\mathcal{R}$, la charge est en mouvement et est donc associée à un vecteur densité de courant :}
$$
	\vec{j}=\begin{cases}
		\rho\vec{v}&\text{si }\vec{r}\in dV\\
		0&\text{sinon}
	\end{cases}
$$

{\txt
Pendant un intervalle de temps $dt$ dans $\mathcal{R}$, la charge se déplace de $dx^\mu=(dt,d\vec{r})$. $dx^\mu$ étant un quadrivecteur, il en est de m\^eme pour $\rho dVdx^\mu$. On considère alors $\rho\frac{cdtdV}{c}\frac{dx^\mu}{dt}$}.

\begin{postulat}
	$cdtdV$ est un invariant.
\end{postulat}

\begin{proof}
	$cdtdV$ étant l'élément d'intégration dans l'espace de Minkovski, on considère la matrice jacobienne.
\begin{rappel}
	Soit $\varphi:(x,y,z)\longrightarrow \vec{\varphi}(x,y,z)=\begin{mat}
		u(x,y,z)\\
		v\\
		w
	\end{mat}$
	
	$$
		\text{Jac}_\varphi=\frac{D(u,v,w)}{D(x,y,z)}=\begin{mat}
			\partial_xu & \partial_yu & \partial_zu\\
			\partial_xv & \partial_yv & \partial_zv\\
			\partial_xw & \partial_yw & \partial_zw
		\end{mat}\\
	$$
	$$
		J_\varphi=\det(\text{Jac}_\varphi)
	$$
	$$
		\text{alors }\bigintssss_{\varphi(x)}f(u_1,...u_n)\,du_1...du_n=\bigintssss_Xf\circ\varphi(x_1,...x_n)|J_\varphi|\,dx_1...dx_n
	$$
\end{rappel}
Pour un boost de Lorentz, on a :
$$
	\begin{mat}
		ct' \\ x' \\ t' \\ z'
	\end{mat}
	=
	\begin{mat}
		\gamma & -\beta\gamma & 0 & 0\\
		-\beta\gamma & \gamma & 0 & 0\\
		0 & 0 & 1 & 0\\
		0 & 0 & 0 & 1
	\end{mat}
	\begin{mat}
		ct \\ x \\ y \\ z
	\end{mat}
	\text{ et par conséquent }J_{boost}=1
$$
$$
	\begin{array}{r@{\;}l}
		\text{d'où\hspace*{0.5cm}} cdtdxdydz&=cdt'dx'dy'dz'\\
		cdtdV&=cdt'dV'
	\end{array}
$$
\end{proof}

\begin{remark}
{\txt Attention aux démonstrations avec $dt'=\gamma dt$ et $dx'=\frac{1}{\gamma}dx$}
\end{remark}

\begin{corol}
	{\txt $cdtdxdydz$ est invariant et donc $\rho\frac{dx^\mu}{dt}$ est un quadrivecteur.}\\
	On pose $J^\mu=\rho\frac{dx^\mu}{dt}$ le quadrivecteur courant (composantes contravariantes).\\
	$J^\mu=(\rho c,\vec{j})=(\rho c,\rho \vec{v})=\rho_0(\gamma c,\gamma\vec{v})$ où $\rho_0$ est la densité de charges dans le référentiel propre de la particule.
\end{corol}

\subsection{Formulation covariante de l'électromagnétisme}
\subsubsection{Retour sur un opérateur}

On a vu que $\nabla$ est un quadrivecteur dont les composantes covariantes sont $\partial_a=\frac{\partial}{\partial{x^a}}$ soit :
$$ 
	\nabla=\left(\frac{1}{c}\frac{\partial}{\partial t},\frac{\partial}{\partial x},\frac{\partial}{\partial y},\frac{\partial}{\partial z}\right)
$$
et ses composantes contravariantes s'écrivent : $\left(\frac{1}{c}\frac{\partial}{\partial t},-\frac{\partial}{\partial x},-\frac{\partial}{\partial y},-\frac{\partial}{\partial z}\right)$\\
$\partial^a\partial_a$ est un tenseur d'ordre 0 obtenu par contraction de deux tenseurs de rang 1 et est donc invariant.
$$
	\begin{array}{r@{\;}l}
			\partial^a\partial_a&=\partial^0\partial_0+\partial^1\partial_1+\partial^2\partial_2+\partial^3\partial_3\\
		&=\frac{1}{c^2}\frac{\partial^2}{\partial t^2}-\frac{\partial^2}{\partial x^2}-\frac{\partial^2}{\partial y^2}-\frac{\partial^2}{\partial z^2} = -\square
	\end{array}	
$$

\subsubsection{Quadrivecteur potentiel}
En jauge de Lorenz ($\nab\cdot\vec{A}+\frac{1}{c^2}\partial_t\phi=0$), les équations satisfaites par les potentiels sont :
$$
	\left\{ \begin{array}{r@{\;}l}
		\nab^2\phi - \frac{1}{c^2}\partial_t^2\phi&=-\frac{\rho}{\epsilon_0}\\
		\nab^2\vec{A}-\frac{1}{c^2}\partial_t^2\vec{A}&=-\mu_0\vec{j}\\
		\square\,\frac{\phi}{c}&=-\frac{\rho}{c\epsilon_0}=-\mu_0\rho c\\
		\square\,\vec{A}&=-\mu_0\vec{j}
	\end{array} \right.
$$
{\txt On est amené à considérer $(\frac{\phi}{c},\vec{A})$ comme les composantes contravariantes d'un quadrivecteur, le quadripotentiel.}
$$
	\square\,\phi^\mu=-\mu_0J^\mu
	\text{ avec } \phi^\mu=(\frac{\phi}{c},\vec{A})
$$
En ce qui concerne la jauge de Lorenz,
$$
	\begin{array}{r@{\;}l}
		0&=\nab\cdot\vec{A}+\frac{1}{c^2}\partial_t\phi=\partial_\mu\phi^\mu\\
		&=\partial_0\phi^0+\partial_1\phi^1+\partial_2\phi^2+\partial_3\phi^3\\
		&=\frac{1}{c}\partial_t\left(\frac{\phi}{c}\right)+\underbrace{\partial_xA_x+\partial_yA_y+\partial_zA_z}_{\nab\cdot\vec{A}}
	\end{array}
$$
La jauge est covariante.

\begin{conc}
	On a été capable de reformuler l'électromagnétique sous forme covariante :
	$$
		\left\{ \begin{array}{r@{\;}l}
			\partial_\mu\phi^\mu&=0\text{\hspace{1cm}(jauge)}\\
			\square\,\phi^\mu&=-\mu_0J^\mu
		\end{array} \right.
	$$
\end{conc}

\subsubsection{Équation de Maxwell-Lorentz}
	Les quadrivecteurs ont quatre composantes, alors que $\vec{E}$ et $\vec{B}$ n'en ont chacun que trois. Il est nécessaire de considérer un tenseur d'ordre supérieur ou égal à 2. On part de l'expression de $\vec{E}$ et $\vec{B}$ avec les potentiels :

$$
	\vec{B}\nab\times\vec{A}\text{\hspace{1cm}et\hspace{1cm}}\vec{E}=-\nab\phi-\partial_t\vec{A}
$$
$$
	\begin{array}{r@{\;}l@{\;}l}
		B_x&=\partial_yA_z-\partial_zA_y=\partial_2\phi^3-\partial_3\phi^2&=-(\partial_3\phi^2-\partial_2\phi^3)\\
		B_y&=&=-(\partial_3\phi^1-\partial_1\phi^3)\\
		B_z&=&=-(\partial_1\phi^2-\partial_2\phi^1)\\	E_x&=-\partial_x\phi-\partial_tA_x=\partial_1\phi^0-\partial_0\phi^1&=-(\partial_0\phi^1-\partial_1\phi^0)\\
		E_y&=&=-(\partial_0\phi^2-\partial_2\phi^0)\\
		E_z&=&=-(\partial_0\phi^3-\partial_3\phi^0)
	\end{array}
$$

{\txt $\frac{E_x}{c}$,$\frac{E_y}{c}$,$\frac{E_z}{c}$,$B_x$,$B_y$,$B_z$ apparaissent donc comme les composantes d'un tenseur d'ordre deux asymétrique :}
$$
	\begin{array}{r@{\;}l}
		F^{\mu\nu}&=\partial^\mu\phi^\nu-\partial^\nu\phi^\mu\\
		F^{\mu\nu}&=\begin{mat}[1.7]
			0 & -\frac{E_x}{c} & -\frac{E_y}{c} & -\frac{E_z}{c}\\
			\frac{E_x}{c} & 0 & -B_z & B_y \\
			\frac{E_y}{c} & B_z & 0 & -B_x \\
			\frac{E_z}{c} & -B_y & B_x & 0 
		\end{mat}
	\end{array}
$$

Les coordonnées covariantes de ce tenseur sont :
$$
	F_{\mu\nu}=g_{\mu\alpha}g_{\nu\beta}F^{\alpha\beta}=\begin{mat}[1.7]
		0 & \frac{E_x}{c} & \frac{E_y}{c} & \frac{E_z}{c}\\
		-\frac{E_x}{c} & 0 & -B_z & B_y \\
		-\frac{E_y}{c} & B_z & 0 & -B_x \\
		-\frac{E_z}{c} & -B_y & B_x & 0
	\end{mat}
$$

On retrouve les équations de Maxwell :
$$
	\nab\cdot\vec{E}=\frac{\rho}{\epsilon_0} \text{\hspace{1.5cm}}
	\begin{array}{r@{\;}l}
		\partial_\mu F^{\mu 0}&=\partial_0F^{00}+\partial_1F^{10}+\partial_2F^{20}+\partial_3F^{30}\\
		&=0+\partial_x\left(\frac{E_x}{c}\right)+\partial_y\left(\frac{E_y}{c}\right)+\partial_z\left(\frac{E_z}{c}\right)\\
		&=\frac{1}{c}\nab\cdot\vec{E}=\frac{\rho}{c\epsilon_0}=\mu_0J^0
	\end{array}
$$
$$
	\nab\times\vec{B}=\mu_0\vec{j}+\mu_0\epsilon_0\partial_t\vec{E} \text{\hspace{1cm}}
	\begin{array}{r@{\;}l}
		\partial_\mu F^{\mu 1}&=\partial_0F^{01}+\partial_1F^{11}+\partial_2F^{21}+\partial_3F^{31}\\
		&=\partial_x\left(-\frac{E_x}{c}\right)+0+\partial_yB_z+\partial_z\left(-B_y\right)\\
		&=\left.\nab\times\vec{B}\right|_x-\frac{1}{c^2}\partial_t\left.\vec{E}\right|_x=\mu_0J_x
	\end{array}
$$

De la m\^eme manière, $\partial_\mu F^{\mu 2}$ et $\partial_\mu F^{\mu 3}$ donnent les composantes $x$ et $y$ de l'équation de Maxwell-Ampère.

Pour les équations de Maxwell avec sources, on a donc :
$$
	\boxed{\partial_\mu F^{\mu\nu}=\mu_0J^\nu}
$$
sous forme covariante. Intéressons-nous aux équations de Maxwell homogènes :
$$
	\begin{array}{r@{\;}l}
		\partial_1F_{23}+\partial_2F_{31}+\partial_3F_{12}&=\partial_x(-B_x)+\partial_y(-B_y)+\partial_z(-B_z)\\
			&= -\nab\cdot\vec{B}\\
			&= 0 \\[10pt]
		\partial_0F_{32}+\partial_3F_{20}+\partial_2F_{03}&=\frac{\partial}{c\partial t}B_x+\partial_z\left(-\frac{E_y}{c}\right)+\partial_y\left(\frac{E_z}{c}\right)\\
			&=\left.\frac{\nab\cdot\vec{E}}{c}\right|_x+\frac{1}{c}\left.\frac{\partial \vec{B}}{\partial t}\right|_x
	\end{array}
$$
M\^eme chose pour $y$ et $z$. Ces équations s'écrivent alors :
$$
	\boxed{\partial_\alpha F_{\beta\gamma} + \partial_\beta F_{\gamma\alpha} + \partial_\gamma F_{\alpha\beta} = 0}
$$

\subsubsection*{Équation de Lorentz}

$$
	\begin{array}{r@{\;}l}
		m\frac{d}{dt}\frac{\vec{v}}{\sqrt{1-\frac{v^2}{c^2}}}&=q(\vec{E}+\vec{v}\times\vec{B})\\
		m\frac{d}{dt}\gamma\vec{v}&=q(\vec{E}+\vec{v}\times\vec{B})\\
		\frac{d}{dt}(\gamma m \vec{v})&=q(\vec{E}+\vec{v}\times\vec{B})\\
		\frac{d}{dt}(\gamma m c^2)&=
	\end{array}
$$
\end{document}