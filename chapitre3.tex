\chapter{Formulation lagrangienne de l'électrodynamique}

\section{Mécanique du point et approche lagrangienne}
{\small \it Note : la première sous-partie se trouve sur une feuille distribuée par le professeur.}
\subsection{Rappel de mécanique analytique classique}
\subsection{Mécanique analytique et relativité}
\subsubsection{Approche lagrangienne}

On a vu (au moins dans le cas de l'électrodynamique) qu'on peut écrire le principe fondamental de la dynamique sous forme covariante :
$$
	\frac{\dif P^\mu}{\dif \tau}=K^\mu
$$
où $K^\mu$ est la quadri-force de Minkowski qui s'écrit en électrodynamique $qF^{\mu\nu}U_{\mu\nu}$. Le but de cette section est de donner une formulation lagrangienne covariante équivalente, \emph{ie.} exhiber un Lagrangien tel que les équations d'Euler-Lagrange soient le principe fondamental de la dynamique.
\begin{itemize}
	\item Difficultés : en mécanique analytique classique, on introduit l'action :
	$$
		S=\bigintssss_{t_1}^{t_2}\dif (\{q_i\},\{\dot{q_i}\},\{t\})\,\dif t
	$$
	et $t$ joue un rôle particulier incompatible avec la relativité. En effet, si $U^\mu=(\gamma c,\gamma \vec{v})$ désigne la quadri-vitesse, $U^\mu U_\mu = c^2$ et les composantes de la quadri-vitesse ne sont pas indépendantes.
	
	\item Retour sur la notion de trajectoire : c'est une courbe paramétrée par un unique paramètre pour lequel la seule contrainte est d'en faire une fonction monotone. En mécanique classique, le choix naturel est $t$. En mécanique relativiste, on introduit un paramètre $\theta$ tel que l'évolution de $\theta$ soit monotone le long de la trajectoire (entendue comme la ligne d'univers suivie par la particule dans l'espace de Minkowski). On impose que $\theta$ soit un invariant de Lorentz. On pourrait prendr
\end{itemize}