\chapter{Formulation lagrangienne de l'électrodynamique}

\section{Mécanique du point et approche lagrangienne}
{\small \it Note : la première sous-partie se trouve sur une feuille distribuée par le professeur.}
\subsection{Rappel de mécanique analytique classique}
\subsection{Mécanique analytique et relativité}
\subsubsection{Approche lagrangienne}

On a vu (au moins dans le cas de l'électrodynamique) qu'on peut écrire le principe fondamental de la dynamique sous forme covariante :
$$
	\frac{\dif P^\mu}{\dif \tau}=K^\mu
$$
où $K^\mu$ est la quadri-force de Minkowski qui s'écrit en électrodynamique $qF^{\mu\nu}U_{\mu\nu}$. Le but de cette section est de donner une formulation lagrangienne covariante équivalente, \emph{ie.} exhiber un Lagrangien tel que les équations d'Euler-Lagrange soient le principe fondamental de la dynamique.

\emph{Difficultés }: en mécanique analytique classique, on introduit l'action :
$$
	S=\bigintssss_{t_1}^{t_2}\dif (\{q_i\},\{\dot{q_i}\},\{t\})\,\dif t
$$
et $t$ joue un rôle particulier incompatible avec la relativité. En effet, si $u^\mu=(\gamma c,\gamma \vec{v})$ désigne la quadri-vitesse, $u^\mu u_\mu = c^2$ et les composantes de la quadri-vitesse ne sont pas indépendantes.
	
{\txt \emph{Retour sur la notion de trajectoire }: c'est une courbe paramétrée par un unique paramètre pour lequel la seule contrainte est d'en faire une fonction monotone. En mécanique classique, le choix naturel est $t$. En mécanique relativiste, on introduit un paramètre $\theta$ tel que l'évolution de $\theta$ soit monotone le long de la trajectoire (entendue comme la ligne d'univers suivie par la particule dans l'espace de Minkowski). On impose que $\theta$ soit un invariant de Lorentz. On pourrait prendre le temps propre $\tau$ mais rien ne nous y oblige, et un tel choix ne permettrait pas de décrire une particule sans masse. À partir de là, on introduit un Lagrangien $\Lambda(x^\mu,\frac{\dif x^\mu}{\dif\theta})$ et une action $\mathcal{S}$ :}
$$
	\mathcal{S}=\bigintsss_{\text{\tiny évènement 1}}^{\text{\tiny évènement 2}} \Lambda(x^\mu,\frac{\dif x^\mu}{\dif\theta})\,\dif\theta
$$
{\txt On pose $\dot{x}=\frac{\dif x}{\dif\theta}$. Il est nécessaire que $S$ soit un invariant de Lorentz pour que la notion d'extrémalisation de $\mathcal{S}$ reste valable quel que soit le référentiel considéré. $\Lambda(x^\mu,\dot{x}^\mu)$ est donc un invariant de Lorentz. Le but est alors de l'expliciter sous forme covariante pour que les équations d'Euler-Lagrange soient équivalentes au principe fondamental de la dynamique :}
$$
	\frac{\dif}{\dif\theta}\left(\frac{\partial \Lambda}{\partial\dot{x}^\mu}\right)-\frac{\partial\Lambda}{\partial x^\mu}=0
$$ 
\begin{remark}
	$\Lambda(x^\mu,\dot{x}^\mu)$ recouvre le cas où le Lagrangien dépend de $t$ explicitement car $x^0=t$.
\end{remark}

{\txt Un choix naturel pour $\theta$ est de considérer le temps propre $\tau$, mais dans ce cas, $\frac{\dif x^\mu}{\dif\tau}\frac{\dif x_\mu}{\dif\tau}=u^\mu u_\mu=c^2$. Pour prendre en compte cette contrainte, on introduit le multiplicateur de Lagrange $\frac{\lambda}{2}$ et on construit :}
$$
	\tilde{\Lambda}=\Lambda(x^\mu,\dot{x}^\mu)+\frac{\lambda}{2}(u^\mu u_\mu-c^2)
$$
et on extrémalise 
$$
	\mathcal{S}=\bigintssss_{\tau_1}^{\tau_2} \tilde{\Lambda}(x^\mu,\dot{x}^\mu)\,\dif\tau
$$
sous contrainte. Les équations de Lagrange s'écrivent alors :
$$
	\left\{ \begin{array}{l}
		\frac{\partial\Lambda}{\partial x^\mu}-\frac{\dif}{\dif\tau}\left(\frac{\partial \Lambda}{\partial\dot{x}^\mu}\right)-\frac{\dif}{\dif\tau}(\lambda u_\mu)=0	\\
		u^\mu u_\mu-c^2=0
	\end{array} \right.
$$
$$
	u^\mu \frac{\dif}{\dif\tau}(\lambda u_\mu)=u^\mu\left(\frac{\partial\Lambda}{\partial x^\mu}-\frac{\dif}{\dif\tau}\left(\frac{\partial \Lambda}{\partial u^\mu}\right)\right)
$$
$$
	\begin{array}{lr@{\;}l}
		\text{D'où}&\frac{\dif \lambda}{\dif\tau}&=u^\mu\left(\frac{\partial\Lambda}{\partial x^\mu}-\frac{\dif}{\dif\tau}\left(\frac{\partial \Lambda}{\partial u^\mu}\right)\right)\\
		\text{Donc}&\lambda &=\bigintsss\dif\tau\,\left( u^\mu \frac{\partial\Lambda}{\partial x^\mu}-u^\mu \frac{\dif}{\dif\tau}\left(\frac{\partial \Lambda}{\partial u^\mu}\right)\right)\\
		&&=\bigintsss\dif\tau\,\left(u^\mu \frac{\partial \Lambda}{\partial u^\mu}+ \frac{\dif u^\mu}{\dif\tau}\frac{\partial\Lambda}{\partial u^\mu}-\frac{\dif u^\mu}{\dif\tau}\frac{\partial\Lambda}{\partial u^\mu}-u^\mu \frac{\dif}{\dif\tau}\left(\frac{\partial \Lambda}{\partial u^\mu}\right) \right)\\
		&&=\bigintsss\dif\tau\,\frac{\dif\Lambda(x^\mu ,u^\mu)}{\dif\tau} - \bigintsss\dif\tau\,\frac{\dif}{\dif\tau}\left(u^\mu\frac{\partial \Lambda}{\partial u^\mu} \right) \\
		&&=\Lambda-u^\nu\frac{\partial \Lambda}{\partial u^\nu}
	\end{array}
$$
Les équations d'Euler-Lagrange deviennent :
$$
	\frac{\partial\Lambda}{\partial x^\mu}-\frac{\dif}{\dif\tau}\left(\frac{\partial \Lambda}{\partial u^\mu}\right)-\frac{\dif}{\dif\tau}\left(\left(\Lambda-u^\mu\frac{\partial\Lambda}{\partial u^\mu}\right) u^\mu\right)=0
$$

Soit $\theta$ un paramètre affine convenable. Alors, pour tout $\sigma \in \mathcal{C}^1$ tel que $\frac{\dif\sigma}{\dif\theta}=\lambda>0$, on impose que :
$$
	\mathcal{S}=\bigintsss_{\theta_1}^{\theta_2}\dif\theta\,\Lambda\!\left(x^\mu,\frac{\dif x^\mu}{\dif\theta}\right) \text{ soit égale à } \bigintsss_{\sigma(\theta_1)}^{\sigma(\theta_2)}\dif\sigma\,\Lambda\!\left(x^\mu,\frac{\dif x^\mu}{\dif\sigma}\right) 
$$
$\mathcal{S}$ est dite \emph{invariante sous une paramétrisation affine}. On a alors :
$$
	\begin{array}{r@{\;}l}
		\mathcal{S}&=\bigintsss_{\theta_1}^{\theta_2}\dif\theta\,\Lambda\!\left(x^\mu,\frac{\dif x^\mu}{\dif\theta}\right)\\
			&=\bigintsss_{\sigma(\theta_1)}^{\sigma(\theta_2)}\dif\sigma\,\Lambda\!\left(x^\mu,\frac{\dif x^\mu}{\dif\sigma}\frac{1}{\frac{\dif\sigma}{\dif\theta}}\right)\\
			&=\bigintsss_{\sigma_1}^{\sigma_2}\dif\sigma\,\Lambda\!\left(x^\mu,\frac{\dif x^\mu}{\dif\sigma}\right)
	\end{array}
$$

{\txt $ \Lambda(x^\mu,\dot{x}^\mu)$ est une fonction homogène du premier degré par rapport à la vitesse, \emph{i.e.} pour tout $\lambda>0, \Lambda(x^\mu,\lambda\dot{x}^\mu)=\lambda\Lambda(x^\mu,\dot{x}^\mu)$.
Le théorème d'Euler assure que $\Lambda(x^\mu,\dot{x}^\mu)=\dot{x}^\mu\frac{\partial\Lambda}{\partial\dot{x}^\mu}$}

Regardons :
$$
	\begin{array}{l@{\;}l}
		\left(\frac{\dif}{\dif\theta}\left(\frac{\partial\Lambda}{\partial\dot{x}^\mu}\right)-\frac{\partial\Lambda}{\partial x^\mu}\right)\dot{x}^\mu&= \frac{\dif}{\dif \theta}\left(\frac{\partial\Lambda}{\partial\dot{x}^\mu}\dot{x}^\mu\right)-\frac{\partial \Lambda}{\partial \dot{x}^\mu}\frac{\dif \dot{x}^\mu}{\dif\theta}-\frac{\partial \Lambda}{\partial x^\mu}\dot{x}^\mu\\
			&=\frac{\dif}{\dif\theta}\left(\frac{\partial\Lambda}{\partial\dot{x}^\mu}\dot{x}^\mu\right)-\frac{\dif\Lambda}{\dif \theta}\\
			&=\frac{\dif}{\dif\theta}\left(\dot{x}^\mu\frac{\partial \Lambda}{\partial \dot{x}^\mu}-\Lambda\right)\\
			&=0
	\end{array}
$$

\begin{conc}
	Modulo la remarque de Dirac (\og  $u^\mu u_\mu-c^2=0$ est une équation faible \fg (\emph{NDR : définition d'une équation faible introuvable sur internet}) \emph{i.e.} on peut faire comme si elle n'existait pas pour les calculs, et s'en souvenir à la fin pour normaliser les résultats), on retrouve les équations d'Euler-Lagrange :
$$
	\frac{\dif}{\dif \theta}\left(\frac{\partial\Lambda}{\partial \dot{x}^\mu}\right)-\frac{\partial\Lambda}{\partial x^\mu}=0
$$
\end{conc}

Il reste à expliciter $\Lambda(x^\mu,\dot{x}^\mu)$. On s'appuit sur les faits suivants :
\begin{itemize}
	\item on doit retrouver le principe fondamental de la dynamique ;
	\item $\Lambda$ doit tendre vers sa formulation non relativiste dans la limite non relativiste ;
	\item $\Lambda$ doit s'écrire sous forme covariante (pour assurer l'invariance de Lorentz). 
\end{itemize}

{\txt Dans le cas d'une particule libre, on sait expliciter le Lagrangien lorsque l'on distingue les conditions temporelle et spatiales, et $\mathcal{L}_{NR}=\frac{1}{2}m\vec{v}^2+\mathrm{cte}$.}
$$
	\begin{array}{r@{\;}l}
		\mathcal{S}=\bigintsss_{t_1}^{t_2} \! \dif t\,\mathcal{L}\!\left(x,t,\frac{\dif x}{\dif t}\right) &= \bigintsss_{\theta_1}^{\theta_2} \!\dif \theta\,\frac{\dif t}{\dif\theta}\mathcal{L}\!\left(x^{\alpha},\frac{\dif x^{\alpha}}{\dif \theta}\frac{\dif\theta}{\dif t}\right)\\
		&\equiv  \bigintsss_{\theta_1}^{\theta_2} \!\dif \theta\,\Lambda\!\left(x^{\mu},\frac{\dif x^{\mu}}{\dif \theta}\right)
	\end{array}
$$
\emph{i.e.} $\frac{\dif t}{\dif\theta}\mathcal{L}\!\left(x^{\alpha},\frac{\dif x^{\alpha}}{\dif \theta}\frac{\dif\theta}{\dif t}\right) =\Lambda\!\left(x^{\mu},\frac{\dif x^{\mu}}{\dif \theta}\right)$ et on choisit $\tau=\theta$, $\frac{\dif t}{\dif \tau}=\gamma$.
$\mathcal{L}$ ne peut dépendre que des vitesses, et $\mathcal{L}=-mc^2\sqrt{1-\frac{v^2}{c^2}}$ est covariant.

$$
	\begin{array}{r@{\;}l}
		\mathcal{L}&\xrightarrow{\text{non relativiste}}-mc^2+\frac{1}{2}m\vec{v}^2=\mathcal{L}_{NR}\\
		\mathcal{L}&=-mc\sqrt{c^2-\vec{v}^2}\\
			&=-\frac{mc}{\gamma}(\gamma^2c^2-\gamma^2\vec{v}^2)^{\frac{1}{2}}\\
			&=\frac{mc}{\gamma}\sqrt{\frac{\dif x^{\alpha}}{\dif\tau}\frac{\dif x_{\alpha}}{\dif\tau}}
	\end{array}$$

Soit finalement : $\Lambda(x^\mu,\dot{x}^\mu)=-mc\sqrt{\frac{\dif x^{\alpha}}{\dif\tau}\frac{\dif x_{\alpha}}{\dif\tau}}$ est covariant.

Retrouve-t-on le principe fondamental de la dynamique ? Les équations d'Euler-Lagrange donnent :
$$
	\begin{array}{r@{\;}l}
		0&=\frac{\dif}{\dif \tau}\left(\frac{\partial\Lambda}{\partial\dot{x}^\mu}\right)-\frac{\partial\Lambda}{\partial x^\mu}\\
			&=\frac{\dif}{\dif\tau}\frac{\partial}{\partial\dot{x}^\mu}\left(-mc\sqrt{g_{\alpha\beta}\dot{x}^{\alpha}\dot{x}^{\beta}}\right)\\
			&=-mc\frac{\dif}{\dif\tau}\left(\frac{1}{2\sqrt{g_{\alpha\beta}\dot{x}^{\alpha}\dot{x}^{\beta}}}( g_{\alpha\beta}\dot{x}^{\beta}\delta_{\alpha\mu}+g_{\alpha\beta}\dot{x}^{\alpha}\delta_{\beta\mu})\right)\\
			&=-mc\frac{\dif}{\dif\tau}\frac{1}{2\sqrt{g_{\alpha\beta}\dot{x}^{\alpha}\dot{x}^{\beta}}}g_{\mu\beta}\dot{x}^{\beta}\\
			&=-mc\frac{\dif}{\dif\tau}\frac{\dot{x}^\mu}{\sqrt{g_{\alpha\beta}\dot{x}^{\alpha}\dot{x}^{\beta}}}
	\end{array}
$$
Soit directement : $\frac{\dif\dot{x^\mu}}{\dif\tau}=0$, en accord avec le principe fondamental de la dynamique $\frac{\dif P^\mu}{\dif\tau}=0$.

\begin{remarks}\hspace*{1pt}
	\begin{itemize}
		\item 
		$$
			\begin{array}{r@{\;}l}
				\dot{x}^\mu\frac{\partial\Lambda}{\partial\dot{x}^\mu}&=\frac{\dot{x}^\mu\dot{x}_\mu}{\sqrt{g_{\alpha\beta}\dot{x}^{\alpha}\dot{x}^{\beta}}}\\
					&=mc\sqrt{g_{\alpha\beta}\dot{x}^{\alpha}\dot{x}^{\beta}}\\
					&=\Lambda
			\end{array}
		$$
		\item Le point clé pour calculer les géodésiques en relativité générale 
		
	\end{itemize}
\end{remarks}